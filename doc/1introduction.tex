\chapter{Introduction}
\section{Domain overview} 
%TODO application types
Wireless Sensor Networks (WSNs) have received large amounts of research the past decades. However this mainly resulted in isolated ad hoc networks. With both the size of WSN's and the amount of networks increasing, the deployment of multiple networks in the same area for different applications made less and less sense. Therefore, recent endeavours have attempted design networks and protocols in order to create a general, ubiquitous internet for automated devices and sensors: the Internet of Things (IoT).

The recent development in IoT has manly focussed on the field of Low Power Wide Area networks (LPWA). These networks serve devices that communicate over large distances with limited computational and communication resources. They therefore entail low data rates, low radio frequencies and raw unprocessed data. These extremely restrictive requirements entail that a regular wireless internet connection does not suffice, as it is not optimized for the extreme resource limitations of LWPA IoT applications.

The scientific progression in the field of IoT has in turn spaked recent commercial interests. Multiple corporations are deveoping and deploying wide area networks for low powered devices. Examples of these networks are Narrow-Band IoT\cite{web:nbiot}, LoRaWAN \cite{web:lora} and Sigfox \cite{web:sigfox}. These networks are deployed and operated by telcom providers and allow instant connectivity by activiating a SIM or network connectivity module. As a consequence large scale LPWA applications are moving from node-hopping and mesh network strategies to operated cell networks \cite{needs source}. Because of the aforementioned reasons the number of connected devices has exploded in the recent years. Estimations vary but a concensus taken from multiple sources predict about 15-30 billion connected devices in 2020. This would imply that by 2020 the number of connected IoT devices will have surpassed the number of consumer electronic devices (e.g. PC's, laptops and phones) \cite{ericson}. 

Both the explosion of devices and entailing explosion of data, and the shift to shared operated cell networks implies a great stress on monitoring sensor applications. While relatively small sized applications on proprietary networks allow for a best-effort approach, the convolution of many large applications on a shared network requires knowledge of the state of the application. The term coined for this is Quality of Service. QoS parameters such as application throughput, service availability and delivery guarantee allow the description of the state of a system or application.

Though the concept of QoS is well understood, there exist challenges in measuring and determining application QoS networks and specifically in WSNs. In the remainder of this introductory chapter we will determine some of these challenges in WSN QoS monitoring and introduce our envisioned method to curtail and combat these challenges. The next [sub]section will deliberate some key obstacles in the current state of art of monitoring  quality of service in wireless sensor networks. After which the succeeding section will introduce our approach to deal with these challenges and capture the QoS in WSN's.
%TODO WSN & IoT interchangeble

\section{Current State of the Art}
As mentioned, some challenges in measuring and determining QoS in WSNs exist. In this [sub]section we will explore three causes of concern in LPWA WSN's and the implications it has on QoS monitoring. We will conclude this section by arguing why the current state of the art does not provide a suitable solution for these challenges.
\subsection{Challenges in monitoring QoS in IoTs and WSNs}
\label{sec:challenges}
Three challenges were identified to cause a disruption of the applicability of current monitoring solutions. These causes will be shortly deliberated individually before summarizing the implications they effect on the domain of QoS monitoring.

\subsubsection{Technical limitations}
The first challenge of LPWA applications is the aforementioned extreme resource constraints. As a LPWA device is required to perform for a certain amount of time (typically at least 10 years \cite{???}) on a finite, bounded battery energy supply, there are no resources to spare for expensive auxiliary processes. Therefore, devices usually send low-level auxiliary data, instead of intelligently derived values. The burden of calculating high level information is therefore deferred to be computed in-network (edge) or in the back-end.

Additionally, evolution of sensor device software is far more restrictive then evolution of server software. Firstly because of the long life-time of devices, it can occur that services based on modern day requirements need to be performed on decade old technology. Secondly, most LWPA networking protocols do not require devices to retain a constant connection, in order to save energy\cite{tmobile, vodafone, iets met nbiot specs}. Instead the devices connect periodically or when an event/interrupt occurs. This entails that devices cannot be updated \emph{en masse}, but individually when a device wakes up. As this requires additional resends of the updated code it consumes more connectivity resources in the network. For this reason LPWA sensor applications often employ a \emph{"dumb sensor, smart back-end"} philosophy. Again deferring the computations to the network or the back-end.

The problem however with deferring the computations further to the back-end is that more and more computations have to be performed centralized. This requires the back-end to be extremely scalable as more jobs need to be performed as more devices are added to the application.

\subsubsection{IoT QoS is different}
Aside from the low-level information sent by the large amout of devices, QoS in WSNs is distinctly different from classical client-server based QoS. Often QoS in a server-based application can be measured at the server. QoS monitoring in a cloud environment may require some aggregation of data, but even then the number of data sources is relatively limited. Large WSN applications require data aggegation by default. As the level of service provided by the application can only be assertained by calculations based on temporal auxiliary data collected from the devices. This concept is known as Collective QoS \cite{collective_qos} and comprises parameters such as collective bandwidth, average throughput and the number of devices that require replacement. As this information eventually requires accumulation on a single machine in order to determine singular values, aggregation of expansive amounts of auxiliary sensor data must be aggregated intelligently as not to form a congestion point or single point of failure.

Alongside of collective QoS we still require device level information. If a device is not performing according to expectations of the predetermined strategy, it is required that this is notified. This introduces a second distinction to classical QoS: multi-level monitoring and reporting. Usually we are only interested in the QoS provided by the sever(s) running our application. However in a wireless sensor environment we require monitoring parameters on different levels. Examples of these monitoring levels are single sensor, the application as a whole or analysis per IoT cell tower or geographic area. This requirement entails data points of different levels of enrichment, calculated from the same raw sensor data.

The final distinction in IoT monitoring is the dynamic nature of WSN applications. An IoT monitoring application needs to be prepared for devices added to the network and dropping out of the application is prone to change of scale and devices are prone to failure and replacement. As a collective QoS parameter is based on a selection of devices, the monitoring application must support adding and remove devices from the equation.

In conclusion IoT QoS management will require a flexible and dynamic method of resource parameter modelling. Additionally this process should be able to be applied on a high influx of sensor date. This monitoring technique is should be able to captivate both lower level (single sensor) and higher level (application) resource distribution.

\subsubsection{Movement to operated cell network}
A final challenge in contemporary QoS monitoring of LPWA applications is the earlier recognised increasing trend of commercial telecom operated cell networks. Though is makes IoT connectivity more efficient because many applications can be served by a single network infrastructure, it does pose some difficulties to QoS. Firstly, Many applications will be competing for a shared scarce amount of network resources. When other applications consume a large portion of the resources, due to poor rationing or event-bursts, your application suffers and cannot provide expected QoS.

Secondly, by out-sourcing the network infrastructure control over the network is lost. Though beneficiary to the required effort, some important capabilities are conceded. For example the network can no longer be easily altered in order to suit the needs of the application. Additionally, auxiliary data can not be extracted from the network and edge computing is not an option, deferring the burden of aggregating QoS data entirely on back-end.

Finally, the telecom operator will require adherence to a Service Level Agreement (SLA). Though this ensures a certain service provided to an application and prevents other applications of consuming extraneous resources, it also requires close monitoring of applications. A breach of the SLA may cause fines or dissolving of a contract. Therefore, strict adherence to the SLA parameters is neccecary and timely proactive intervention is required, if the limits of the SLA are threatened to be exceeded. \cite{zoek refs in rtopics}

To summarize, outsourcing the management of the network infrastructure to a professional telecom provider aggravates the need for exact and real-time curtailment of digital resources, while simultaneously impeding our ability to do so in the network itself. We will need to remedy this by adapting the parts of the network architecture we do control, i.e. the sensor devices and the back-end application. Because of earlier proposed concerns and challenges, this increased responsibility will be mostly attributed to the back-end application.

\subsubsection{Summation}
In conclusion, The tendency to defer computations challenges the computing capabilities of centralized solutions. This inability for pre-computation, combined with immense input numbers of LPWA device data, entails a design with a deliberate focus on scalability of throughput. Additionally measuring and controlling QoS in Wireless Sensor networks is very different from measuring and controlling QoS in resource-abundant networks. Both because of the resource constraints and the fact that the QoS characteristics in WSN's are different from the characteristics in conventional networks and applications. Finally, by outsourcing the responsibility of network management, the ability to control and observe those networks is also lost.

To this purpose we will research the applicability and design of a WSN QoS platform. This platform should address the issues of scalability and limitations of source devices and in-network processing. It should be noted however that we will not address the issues of end-device resource restriction and network obscuration directly, only the challenges it imposes on the task of QoS monitoring and control.

\subsection{Deficiencies in current State of Art}
Several platforms exist that are capable monitoring and controlling IoT applications to some degree \cite{refs_to_platoforms}. However all are lacking in some of the important considerations. These platforms are either not conceived with a focus on LPWA's severe resource constraints, a primary focus on resource and QoS monitoring or the extreme scale of contemporary WSN applications \cite{platform surveys}.

These deficiencies make the existing monitoring platforms insufficient solutions for monitoring and controlling large scale LPWA IoT applications. This implies that the technologies are either inapplicable or require a composition of these technologies. This complication of the technology stack would be acceptable for a key function of an application, but not for an auxiliary monitoring processes. As to not complicate a software product which does not enjoy the main focus of development efforts it would be beneficiary to have a single platform which enables it's development.

%TODO uitbreiden
[TODO] uitbreiden

\section{Goal}
\label{sec:goal}
The goal of this study is to research and develop a single development platform capable of measuring and monitoring QoS parameters of LWPA WSN's. This platform will be devised to overcome the challenges identified in the previous section (Section \ref{sec:challenges}). To re-iterate, these core challenges are: the deference of processing to the back-end, due to restricted processor capabilities and obscuration of the network, and the unique QoS challenges in WSN networks such as multi-level abstractions and aggregation of massive amounts of multi-sourced snapshots. The platform to be designed will enable development of support applications that process auxiliary IoT data. This data is raw and low-level, but is enriched by the platform by associating streaming data with data obtained from relevant data sources and aggregating streaming data to infer higher-level information. this information can be exported for reporting and visualization purposes, can alter the state of a system (single sensor, group of sensors, entire application, etc.) and can cause alerts to be dispatched for immediate intervention.

\subsection{Research questions}
To accomplish the goal set out for this study the following question require answering.
\begin{enumerate}[leftmargin=24pt, label=\small RQ\arabic*]
\nospace
\item\label{rq:identify_processes} What are the key data transformations and operations that are performed on (auxiliary) data streams generated by WSNs?
\item\label{rq:desing_processes} How to design a platform that facilitates the identified WSN data streams, transactions and operations?
\item\label{rq:abstraction} What is the appropriate level of abstraction for a WSN monitoring platform, such that
\begin{itemize}
\nospace
\item the platform is applicable to monitoring a large domain of WSNs, and
\item allows for the highest ease-of-implementation?
\end{itemize}
\item\label{rq:identify_scale} What are the challenges regarding scalability in a WSN data stream processing platform?
\item\label{rq:design_scale} How can these challenges be overcome?
\item\label{rq:idenfity_model} What are the key concepts regarding modelling and calculation of QoS parameters?
\item\label{rq:design_model} How can we model the state of a system with variable behaviour?
\item\label{rq:solve_model} How can we determine the optimal system behaviour in accordance with its state?
\end{enumerate}

From the listed research questions we find a focus that is twofold. The first point of focus is the composition and development of an abstract, scalable streaming platform for IoT data enrichment. The associative questions are RQ1-5. It concerns the appropriate abstraction of a platform combatting the challenges in iteratively refining low-level sensor data to high-level information with business value and scalability due to the vast amount of data generated by the WSN.  The second focal point concerns the representation and processing of information depicting the state of a system under investigation. This entails capturing some data points produced by sensor devices or intermediary processes, calculating the derived parameters from those measurements and producing a decision in accordance with the model's values and set rules.

\section{Approach}
With the goal and research questions defined, we will clarify the method we aim to employ to complete this goal.

As the above section mentioned the research questions can be divided into two categories: The design of the platform and modelling the distribution of resources and QoS parameters. Our approach is therefore to research these individually before integrating these efforts into one resulting software development platform. First we will explore the design of a processing platform architecture that endeavours to compete the challenge of immense input of data. Additionally, it will feature multi-stage calculation and enrichment in order to provide for the need of multi-level QoS processing and reporting. Consequently, we will research a modelling method capable of captivating the distribution of resources and interconnectivity of quality of service. This model will again take into account the multi-level modelling needs in accordance with the identified challenge. Additionally it will combat the challenge of enriching deferred low-level data to high level usable information by allowing transformations of resource parameters. 

Each point of focus will be devised, designed and developed according to the following schedule. We will first explore the problem domain of the to be designed solution/model. This will be performed with a commonality/variability analysis (Section \ref{sec:back:cv_analysis}). This analysis allows us to conceptualize the problem domain which will determine the appropriate level of abstraction for both the platform architecture and the resource distribution model. This analysis will result in a list of requirements for the solution to adhere to. With the requirements defined the state of the art of the problem domain will be explored to identify viable technologies and their deficiencies, before selecting the best applicable technologies. With these technologies identified we will adapt, design and develop the intended artifact. For design and development we will adopt the iterative development approach of Design Science Methodology\cite{dsm} (section \ref{sec:back:dsm}). Ultimately, the devised solution will be evaluated and discussed by paralleling them to the set requirements and some additional concepts and criteria. 

Finally the conceived model will be incorporated in the larger context of the developed platform architecture. Once the two compounds have been integrated into a single solution, the challenges it claims to combat will need verifying. In order to perform initial validation of the developed solution it will be applied to a real-world commercial car park WSN application developed and maintained by the Dutch company Nedap N.V. This will be achieved by providing a prototype implementation of the constructed platform. The development efforts and applicability of the development platform will be evaluated according to preconceived metrics and conditions.
%TODO diclaimer partial validation


\section{Organisation of thesis}
%TODO finally do
[TODO]