\documentclass[a4paper, 10pt, conference]{report}
\usepackage{titlesec}
\usepackage{enumitem}
\usepackage{hyperref}
\usepackage{graphicx}
\usepackage{rotating}
\usepackage{float}
\usepackage{units}
\usepackage{pifont}
\usepackage{enumitem}
\usepackage{qtree}
\usepackage[export]{adjustbox}
\usepackage[numbers,sort]{natbib}
\usepackage{listings}
\titleformat{\chapter}{\normalfont\huge}{\thechapter.}{20pt}{\huge\textbf}
\newcommand{\cmark}{\ding{51}}%
\newcommand{\xmark}{\ding{53}}%
\newcommand{\nospace}{\setlength\itemsep{0em}}
\begin{document}
\author{J.J. Schutte, University of Twente\\\emph{j.j.schutte@student.utwente.nl}}
\date{June 27, 2019 \\ \vspace{26em} \includegraphics[width=\textwidth ,center]{resources/img/logos.png}}
\title{\textbf{Master Thesis} \\Design of a development platform to monitor and manage Low Power, Wide Area WSNs
}
\maketitle
%\begin{titlepage}
%\large
%\centering
%{\Huge{Master Thesis}\\}
%{\Large{Design of a development platform to monitor and manage Low Power, Wide Area WSNs}} \\

%J.J. Schutte, University of Twente \\
%\emph{j.j.schutte@student.utwente.nl} \\
%June 27, 2019
%\end{titlepage}


\abstract{The recent explosion of Low Power Wide Area (LPWA) WSN devices has raised interest in perceiving the Quality of Service (QoS) provided to and by such applications. Current QoS solutions do not respect LPWA-specific considerations, such as limited resources and extreme scale. This study has set out to research an appropriate solution to QoS monitoring and management that does concern these considerations. This is achieved by establishing a development platform focused on LPWA QoS. The platform consists of two chief concepts. The first of which is a distributed stream processing architecture. The architecture back-bone is based on Apache Storm and provides scaffolding for different classes of stream transformations, which guides users in implementing their monitoring applications. The second artefact is a model capable of captivating resources and calculating the performance of a system, considering different modes of operation of that system. The proposed development platform is validated by implementing an instantiation of it, based on an actual, commercial on-street parking application. Though the study shows some deficiencies still present in the solution, its results demonstrate it as an applicable and feasible aid in constructing scalable applications capable of QoS monitoring in LPWA WSNs. \\ \\
%{Keywords --- WSN, IoT, LPWA, QoS}
\textbf{Keywords}: Wireless Sensor Networks, Internet of Things, LPWA, Quality of Service
\subsubsection*{Committee}
This thesis was supervised and examined by:
\begin{table}[h]
\begin{tabular}{ll}
prof.dr.ir. M. Aksit & University of Twente, Formal Methods and Tools \\
dr. N. Meratnia & University of Twente, Pervasive Systems\\
R. Boland & Nedap N.V., Identification Systems \\
\end{tabular}
\end{table}
%\item[prof.dr.ir. M. Aksit] Formal Methods and Tools - University of Twente
%\item[dr. N. Meratnia] Pervasive Systems - University of Twente
%\item[R. Boland] Nedap N.V.
\newpage
\tableofcontents
\newpage
\chapter{Introduction}
\section{overview} \\
\section{Problem statement}
\section{State of the art (working title)}
Current state of affairs \\
deficit in current state of affairs
\section{Approach}
\subsection{Goal}
\subsection{Reserch questions}
\subsection{General methodology}
\section{Organisation of thesis}

\chapter{Backgound}
\section{Sensor networks/IoT}
\section{Micro-services}
\section{Design Science methodology}
more to add as needed.

%\setcounter{chapter}{4}
%\setcounter{page}{54}
\newcommand{\archid}{1}
\chapter{Design of IoT monitoring platform architecture}
\label{ch:architecture}
%What needs to be monitored: sensor-networks/IoT 
%How data and information streams flow and combine 
%will focus on data sources and streams as to ascertain which architecture, configuration and components are needed/suitable.
This chapter will detail the process taken in order to device the platform and its architecture. This will be accomplished by first exploring the general problem domain. Subsequently the design of the proposed platform and its implementation will be deliberated by identifying the available supporting technologies, clarifying the adaptations made to those technologies and explaining further implementation details. The chapter will be concluded by discussing the advantages, limitations and considerations of the proposed solution.
\section{Goal}
Large sensor applications send immense amounts of low-level raw monitoring data that requires capturing and enriching. Individual snapshots of raw data might contain very little information. However when accumulated, these snapshots contain the potential from which meaningful conclusions can be derived. These decision range from single sensor scale to the sensor application as a whole. This raw data is enriched by combining and analysing datasets of similar, related data, in order to achieve a higher degree of information. The goal of the efforts described in this chapter is to conceive a software platform that enables software developers to construct their own sensor application monitoring system. The intention to achieve this is by devising a generic application backbone and base building blocks for  developers to extend and compose.  
\section{Conceptualization of the problem domain}
In this section the problem domain will be investigated in order to eventually determine the requirements for the model. This will achieved by performing a commonality/variability analysis (C/V analysis) of the problem domain, as described in section \ref{sec:back:cv_analysis}. The analysis consists of three concepts:
\begin{itemize}
\nospace
\item The definitions that will be used in the analysis and the remainder of this chapter, 
\item the common features shared by all elements in the problem domain and which may be assumed as established concepts, and 
\item the variations that appear between aspects of the problem domain for which must be accounted for in the proposed solution.
\end{itemize}
\subsubsection*{Definitions}
Firstly, some key terms will be defined that will be used in the analysis and the remainder of this chapter.
\begin{description}[style=nextline]
\nospace
\item[Platform] The monitoring platform to be designed.
\item[Application] The application that is being investigated by the platform.
\item[Snapshot] A message containing a collection of data-points indicating the state of a system on a certain instant.
\item[Source] An entity emitting a snapshot. This can be a physical end-device, external service or an process internal to the platform.
\item[Consequence] An action effected by the platform based on the analysis of one or more snapshots.
\end{description}
\subsubsection*{Commonalities}
With the definitions established some common features shared by each application in the problem domain will be identified next. These commonalities may be presumed during the design of the platform and grants a scope to the design efforts.
\begin{enumerate}[label=C\archid .\arabic*]
\nospace
\item \label{c:scale_sensor} The group of target applications involves a huge amount of sensors ([scale] which entails a high throughput of snapshots requiring analysis by the platform.
\item \label{c:snapshot} As mentioned in the definitions, data is captured in snapshots. These represent the (partial) state of the application as measured or determined at a certain point in time. These snapshots can be used for both input of the platform as for representing intermediary computation states.
\item \label{c:snapshot_transformation} The parameters and values of a snapshot, and therefore consecutive derived values, may be considered fixed. Parameters can only change by outputting a new snapshot, not during evaluation of the current one.
\end{enumerate}
\subsubsection*{Variabilities}
Finally, the variety within the problem domain will be explored. As the purpose of the solution is to process information the analysis will mostly focus on the variations in the domain of data and information produced by applications. The solution should provide proficient adaptability in order to account for this variability. This will be ensured by captivating these variations in requirements.
\begin{enumerate}[label=V\archid .\arabic*]
\nospace
\item \label{v:qoi} The first variety  encountered is the variation in Quality of Information (QoI). As described in the Background chapter (section \ref{sec:back:qoi}) there are many parameters characterizing the QoI of data. QoI can vary on any combination of them.
\item \label{v:conclusion_basis} Secondly, there is the information base on which conclusions are made. The first conclusion basis is elementary:
\begin{enumerate}
\nospace
\item Single snapshot. (e.g. a sensor requiring maintenance)
\end{enumerate}
The second identified analysis is based on a large amount of low-information snapshots \cite{qos_difficult}, of which two types are identified:
\begin{enumerate}[resume]
\item Multiple sequentially relevant snapshots from a single source (longitudinal), used to analyse tendency of parameters. (e.g. a sharp continuous increase in bandwidth used which may imply future capacity issues.)
\item Many multi-source snapshots without individual significance (lateral). E.g: while the individual bandwidth usage of sensors may be of little interest, knowledge of the average and total bandwidth usage of the system may be warranted.
\end{enumerate}
\item \label{v:consequence} The possible consequences by the platform have a large range of implementations and cannot be fully anticipated. However, though the exact implementation of consequences can never be anticipated exactly, some groups of consequences can be identified.
\begin{enumerate}
\nospace
\item Build a model for reporting purposes. In order to generate reports some high-level information data-points need to be calculated based on large datasets. these data-points are then exposed either by an in-memory component with an API or by persisting it to intermediary permanent storage.
\item Analyses which invoke immediate responses to the application or a command \& control service administrating the application.
\item Alerting or reporting according to a specified rule. When this user defined rule is met or violated an alert is sent to a maintenance operator or auxiliary system.
\end{enumerate}
\end{enumerate}
The final variety is the scale of the application. It has already been established that the platform will operate on applications of very large scale, i.e. thousands of sensors. However, given a thousand as lower bound, the upper bound is still uncertain. Therefore the size of the application is still uncertain and differing degrees of size require different computational needs.
\begin{enumerate}[label=V\archid .\arabic* , resume]
\nospace
\item \label{v:scale} The scale of large wireless sensor applications varies wildly. This yields for both the number of devices in the application and the rate at which the devices emit data.
\end{enumerate}
\section{Requirements for the proposed software platform}
In this section the requirements of the proposed platform will be described, in accordance with the variability identified in the previous section. 
\begin{enumerate}[label=R\archid .\arabic*]
\nospace
\item \label{r:snaptshot_transformation} The platform should enable the capture and transformation of snapshots.
\item \label{r:basis_single} The platform should enable processing of single snapshot.
\item \label{r:basis_historic} The platform should enable processing of a window of homogeneous snapshots.
\item \label{r:basis_accumulated} The platform should enable processing and aggregation of an enormous amount of snapshots.
\item \label{r:consequence} The platform should enable implementation of a wide range of consequences. It should at least provide for these anticipated types of consequence:
\begin{itemize}
\nospace
\item report building
\item application feedback
\item alerts of behavioural violations
\end{itemize}
\item \label{r:scale} the platform should be scalable in order to support any large amount of inputs
\end{enumerate}

\subsubsection*{Justification}
This section will be concluded by justifying the identified requirements according to the earlier performed C/V analysis. A formal traceability between the requirements, commonalities and variability is listed in table \ref{table:3_justification}

\begin{table}[H]
\centering
\begin{tabular}{|l|l|l|} \hline
Requirement & Variability &  Commonality \\ \hline
\ref{r:snaptshot_transformation} & \ref{v:qoi} & \ref{c:snapshot}, \ref{c:snapshot_transformation}\\ \hline
\ref{r:basis_single} & \ref{v:conclusion_basis}a & \\ \hline
\ref{r:basis_historic} & \ref{v:conclusion_basis}b & \\ \hline
\ref{r:basis_accumulated} & \ref{v:conclusion_basis}c & \\ \hline
\ref{r:consequence} & \ref{v:consequence} & \\ \hline
\ref{r:scale} & \ref{v:scale} & \ref{c:scale_sensor} \\ \hline
\end{tabular}
\caption{traceability table for justification of requirements}
\label{table:3_justification}
\end{table}

The first requirement (\ref{r:snaptshot_transformation}) regards the definition and concepts of snapshots and is based on the commonalities and the variation in Quality of Information (Section \ref{sec:back:qoi}). As illustrated by the traceability table, the following three requirements (\ref{r:basis_single}--\ref{r:basis_accumulated}) closely correlate with the three varieties identified in \ref{v:conclusion_basis}. Requirement \ref{r:consequence} attempts to captivate the variability described in \ref{v:consequence}. This variation is captured in a single requirement as opposed to differentiating them as for \ref{v:conclusion_basis}. This is because the possible consequences are not limited to the identified consequence groups. Therefore, they are grouped into one abstract requirement. Lastly, the final requirement considers the scale of the target applications. This regards both the amount of devices in the target application as the frequency the send their snapshots. 

\section{Exploration of the solution domain}
This section will explore the solutions and supporting technologies that are offered. First, the  base architecture type will be considered of the platform, as it is the most fundamental decision to be made. Continuing, options for supporting technology will be explored. The section will be concluded by examining some distributed computing technologies. These technologies should enable data-intensive computations by distributing them over a cluster, as to provide the required scalability.
\subsection{Architecture and execution platform}
\subsubsection*{High level architecture}
The first decision to make is the high-level architecture to adopt. The first option for which is to implement the platform as a monolithic software system. The benefit of such a system is that it keeps the solution as simple as can be. This is reflected by a famous proverb of Edsger Dijkstra: ``Simplicity is a prerequisite for reliability'' \cite{dijkstra}. This simplicity entails a better understanding of the product by any future contributor or user, without the need to consult complex, detailed documentation. However monolithic software products have been known to be difficult to maintain, because code evolution becomes more difficult as development progresses and changes and additions are made to the code base. Additionally, monolithic software systems are notoriously difficult to scale and balance \cite{mono_vs_micro}, which violates requirement \ref{r:scale}. Therefore, instead distributed micro-component approach will be adopted.

Converse to the monolith is the micro-component architecture. It consists of a multitude of smaller components that are functionally distinct. These components communicate to one another through a underlying message distribution system. By functionally encapsulating the application into distinct modules, an inherent separation of concerns is achieved. This in turn reduces entangulation and improves the application's capacity for evolution. Micro-components are  more flexible than monoliths, allow for better functional composition, are easier to maintain and much more scalable \cite{mono_vs_micro}. Additionally, distributed cloud computing solves some of the tenacious obstacles in IoT's, such as the constraint computational and storage capacity \cite{benefits_cloud_to_iot}. The remainder of this section will investigate some established technologies enabling micro-service architectures.

\subsubsection*{Apache Storm}
Apache Storm is a big-data streaming library especially designed for separation of concerns and scalability. It achieves distributed compution by partitioning the stages of computation. It separates stages of computation in distinct processors performing a portion of the global process. These processors are composed into a topology. This topology specifies which processors communicate to which other processors using Storm's inherent message broker. By breaking up the computation, different stages can be distributed among machines and duplicated if required. The Storm topology consists of three chief concepts.
\begin{description}[style=nextline]
\nospace
\item[Spouts] Nodes that introduce data in the topology,
\item[Bolts] Nodes that perform some computation or transformation on data, and
\item[Topology] An application-level specification of how nodes are connected and messages distributed.
\end{description}
The computation is regarded as a directed graph with spouts and bolts as vertices, and an internalized messaging system as edges.

%By breaking up the computations into multiple consecutive bolts, Storm allows computations to be spread over a cluster. Additionally Storm allows individual bolts to be replicated and executed separately. This lateral distribution reduces the occurrence of bottlenecks in the network due to bolts executing expensive pivotal processes

Storm is especially suited for the purpose of this study since it was designed for interconnected micro-components. By employing Apache Storm both the distributed computation environment as the means of data distribution are obtained, simplifying the technology stack.

However, the built-in stream distribution mechanism is completely internalized, complicating integration with auxiliary processes. Tasks such as data injection, platform monitoring and data extraction for debugging, processing or reporting by third-party programs and stakeholders will require an exposing mechanism. Additionally, Storm requires bolt connections to be explicitly defined at start-up. This causes two disadvantages: Firstly, a single process cannot be updated or reconfigured without restarting the entire topology. Considerations should therefore be made on when to update the system and when to delay rolling-out an updated version. Secondly, the bolts are connected pair-wise. This is in contrast to most conventional publish/subscribe communication platforms (such as Kafka and RabbidMQ). These systems decouple the producer and consumers and instead write and read to addressable communication channels (topics). Storm allows reading and listening on streams of a certain topic, but the connection still needs to be explicitly specified. This is cumbersome, but should be able to be overcome. 
%Though cumbersome, this also grants an advantage. With strong component bindings it should prove more difficult to deploy an invalid architecture due to small mistakes such as mistypes or not updating all topic bindings on a refactor. 

\subsubsection*{Micro-component architecture without execution platform}
A final option is to employ a micro-component architecture without an execution platform. Instead, this requires distributing components manually and have them communicate using message brokers. This would increase the efforts required to develop and deploy the platform, but does provide greater control over its execution. Additionally this would alleviate the deficiencies identified for Apache Storm, such as difficult third party integration, cumbersome topology building and lack of run-time reconfiguration. 
%TODO runtime environment

\subsection{Message brokers}
%\subsubsection{Native HTTP}
%The Hypertext Transport Protocol (HTTP) \cite{def:http} is a [onmiskenbaar] communication standard this is widely employed in internet communications. It is well-documented and familiar to almost every industry professional, which should [ease] implementation and maintenance. Aside of maintainability HTTP is very versitile, which should ensure that it meets the needs for our system. However, this versitiliy stems from the barebone definition of the protocol. [TODO afmaken]. Additionally, HTTP routing is performed based on the IP address and port of the target process. This requires any sending component to know all its listener components, requiring either direct IP-based subscription or a discovery/lookup service. 
By employing a micro-component architecture (without inherent messaging system), a communication technology for components to communicate to each other needs to identified. This approach employs a service to which producers write messages to a certain topic. Consumers can subscribe to a topic and subsequently read from it. This obscures host discovery, since a producer need not know its consumers or vice versa. The routing is instead performed by the message service. The following will explore the two widely used message broker services in the industry.

\subsubsection*{RabbidMQ}
RabbitMQ \cite{web:rabbitmq} is a distributed open-source message broker implementation based on the Advance Message Queue Protocol. It performs topic routing by sending a message to an exchange server. This exchange reroutes the message to a server that contains the queue for that topic. A consumer subscribed to that topic can then retrieve it by popping it from the queue. Finally, an ACK is returned to the producer indicating that the message was consumed. The decoupling of exchange routers and message queues allows for custom routing protocols, making it a versatile solution. RabbitMQ operates on the \emph{competing consumers} principle, which entails that only the first consumer to pop the message from the queue will be able to consume it. This results in an \emph{exactly once} guarantee for message consumption. This makes it ideal for load-balanced micro-component applications, because it guarantees that a deployment of identical services will only process the message once. It does however make multi-casting a message to multiple types of consumers difficult.

\subsubsection*{Apache Kafka}
Converse, Apache Kafka \cite{web:kafka} distributes the queues itself. Each host in the cluster hosts any number of partitions of a topic. Producers then write to a particular partition of the topic, while consumers will receive the messages from all partitions of a topic. Because a topic is not required to reside on a single host, it allows load balancing of individual topics. This does however cause some QoS guarantees to be dropped. For instance, message order retention can no longer be guaranteed for the entire topic, but only for individual partitions. Kafka, in contrast to RabbidMQ's competing consumers, operates on the \emph{co-operating consumers} principle. It performs this by, instead of popping the head of the queue, a pointer is retained for each individual consumer. This allows multiple consumers to read the same message from a queue, even at different rates. The topic partition retains a message for some time or maximum number of messages in the topic, allowing consumers to read a message more then once. Ensuring that load-balanced processes only process a message once is also imposed on the consumer by introducing the notion of consumer groups. These groups share a common topic pointer, which ensures that the group collectively only consumes a message once. This process does not require an exchange service, so Kafka does not employ one. This removes some customization of the platform, but does reduce some latency. Lastly, Kafka does not feature application level acknowledgement, meaning that the producer cannot perceive whether its messages are consumed.

\subsubsection*{Comparison}
\begin{table}
\centering
\begin{tabular}{|l||c|c|}\hline
  					& RabbidMQ 			& Kafka 		\\ \hline 
Speed				& + 				& ++ 			\\ \hline
Scalable			& +					& ++	 		\\ \hline
Multi-cast			&\xmark				& \cmark		\\ \hline
Multiple reads		&\xmark				& \cmark		\\ \hline
Acknowledged		&\cmark				& \xmark		\\ \hline
Delivery guarantee	&\cmark				& \xmark		\\ \hline
Consumer groups		&\cmark				&\cmark			\\ \hline
Retain ordering		&Topic level		& Partition level\\ \hline
Consumer model	 	&Competing			& Cooperating	\\ \hline
\end{tabular}
\caption{Summary comparison of RabbidMQ and Kafka}
\label{table:rabbidmq-kafka}
\end{table}

 

A comparative summery of both technologies is given in table \ref{table:rabbidmq-kafka}. From this comparison the first apparent difference is the approach taken to consumer strategies. Kafka allows messages to be read multiple times, both by different consumers or the same consumer, whereas RabbidMQ allows messages to be consumed only once. Secondly, Kafka's lower-level replication provides increased scalability and speed \cite{kafka_vs_rabbitmq}. However, it does so at the cost of some functional benefits such as order retention, guaranteed delivery.

%Following this comparason we have chosen to employ Kafka for our platform. The first observation is that Kafka performs better in non-functional metrics. Sources report Kafka to be 2--4 times faster than RabbidMQ \cite{speed_kafka} and the partitioned topics allow Kafka to be distribute and scale overloaded channels. Secondly, the cooperating consumer model Kafka is based on allows us to natively multicast messages to multiple consumers, while still being scalable by defining consumer groups. By choosing for Kafka we do however default some features such as producer acknowledgement and topic level order guarantees. As for producer acknowledgement we do not require it, as producers simply send messages into the clear and consumers are required to make efforts that it processes all data. Using the feature to read messages more than once, we should be able to build a dependable platform. Finally, Kafka cannot guarantee the read order of partitioned topics. We therefore will need to enforce it ourselves in the platform and implementations of it. This can be either done by sorting messages in buffers on some ordered parameter (e.g. timestamp or sequence number) or by not partitioning topics containing order-critical streams.

\subsection{Distributed computing}
As specified by requirement \ref{r:basis_accumulated}, a means of processing large volumes of data is required. This is accomplished by aggregating a large number of snapshots into a distinct smaller amount of snapshots (often singular) with a higher-degree of information. In order to accomplish this a scalable means of computation is required (requirement \ref{r:scale}) 
\subsubsection{MapReduce}
MapReduce \cite{mapreduce} is a distributed computing framework. It operates by calling a \emph{mapper} function on each element in the dataset, outputting a set of key-value tuples for each entry. All tuples are then reordered, grouped as sets of tuples with a common key. The key-value sets are then distributed across machines and a \emph{reduce} function is called to reduce the many individual values into some accumulated data-points. The benefit of this framework is that the user need only implement the \emph{mapper} and \emph{reduce} functions. All other procedures, including tuple distribution and calling the mapper and reducer, are handled by the framework. An example of the algorithm on the WordCount problem is illustrated in Figure \ref{img:mapreduce}.

The concept of a mapped processor is of a large benefit to the platform. In the early exploration phase it quickly became apparent that there were many use cases where one might want to extract accumulated snapshots per individual sensor or grouped by cell tower. This approach also allows to compensate for devices sending at different rates. These devices would be overrepresented in the population if they were not normalized. By first grouping and averaging the messages per device, it can assured that every device has the same weight in the analysis.

\begin{figure}
\centering
\includegraphics[width=\textwidth]{resources/img/mapreduce.png}
\caption{The overall MapReduce word count process \cite{mapreduce_img}}
%https://www.todaysoftmag.com/article/1358/hadoop-mapreduce-deep-diving-and-tuning
\label{img:mapreduce}
\end{figure}

Though the ease of implementation is very high and the technology is very applicable to the platform, the algorithm has proved to be comparatively slow. The reason for this is that before and after both the map and reduce phase the data has to be written to a distributed file system. Therefore though highly scalable, the approach suffers by slow disk writes \cite{mapreduce_vs_spark}. Finally, MapReduce works on large finite datasets. Therefore, the stream data must be processed into batches in order for MapReduce to be applicable.

\subsubsection{Apache Spark (Streaming)}
Apache spark \cite{web:spark} is an implementation of the Resilient Distributed Dataset (RDD) paradigm. It employs a master node which partitions large datasets and distributes it among its slave nodes, along with instructions to be performed on individual data entries. Operations resemble the functions and methods of the Java Stream package \cite{java_stream}. 

Three sort of operations exist: narrow transformations, wide transformations and actions. \emph{Narrow transformations} are parallel operations that effect individual entries in the dataset and result in a new RDD, with the original RDD and target RDD partitioned equally. Examples of such functions are \emph{map} and \emph{filter}. Because these transformations are applied in parallel and partitioning remains identical, many of these transformations can be performed sequentially without data redistribution or recalling the data to the master. \emph{Wide transformations} similarly are applied on individual dataset entries, but the target RDD may not be partitioned equal to the original RDD. An example of such a transformation is \emph{groupByKey}. Since elements with  he same key must reside in the same partition, the RDD might require reshuffling in order for computation to complete. Finally, Actions, such as \emph{collect} and \emph{count} require the data to be recalled to the master and final calculation is performed locally, resulting in a concrete return value of the process. RDD's provide an efficient distributed processing of large datasets, that is easy to write and read. However careful consideration must be given to the operations and execution chain in order to avoid superfluous dataset redistribution \cite{spark_programming_guide}.

\begin{small}
\vspace{8px}\hrule
\begin{lstlisting}[
language=Java,
caption=MapReduce example of Figure \ref{img:mapreduce} in Spark RDD.,
captionpos=b, 
escapeinside={(*}{*)}, 
columns=flexible,
numbers=left,
tabsize=4,
breaklines=true,
label=list:mapreduce_spark-search
]
// assumes initial RDD with lines of words = lines
JavaRDD<String[]> wrdArr = 				lines.map(l->l.split(" "));
JavaRDD<String> words =					wrdArr.flatMap(arr -> Arrays.toList(arr));
JavaRDD<String, Integer> pairs =		words.mapToPair(x->(x,1));
JavaRDD<String, Integer> counts = 		pairs.reduceByKey((a,b) -> a+b);
Map<String, Integer> result =			counts.collectAsMap();(*\vspace{5px}\hrule*)
\end{lstlisting}
\end{small}

It is interesting to note that the MapReduce framework can easily be reproduced in Spark. this is achieved by calling the \emph{map} and \emph{reduceByKey} subsequently. To illustrate this the MapReduce procedure of Figure \ref{img:mapreduce} is implemented using Apache Spark in Listing \ref{list:mapreduce_spark-search}. Please note that the intermediate assignments of the RDD are not required. RDD operations can be chained after one another, but intermediate assignments have been used to better illustrate the steps taken. Also note that the first three steps are be performed fully parallized since they are all narrow transformations. Only line 5 (wide transformation) and 6 (action) require RDD redistribution.
% ref e.g.: https://jaceklaskowski.gitbooks.io/mastering-apache-spark/spark-rdd-transformations.html

Additionally, the framework does not require disk writes as MapReduce does. Instead, it runs distributed calculations in-memory, thereby vastly improving the overall calculation speed. This does however raises a reliability issue, because if a slave node fails, its state cannot be recovered. Such occurrences are resolved by the master by replicating the part of the dataset from the intermediate result it retained and distributing it among the remaining slave nodes. Because the sequence of transformations is deterministically applied to each individual entry in the dataset any new slave node can continue calculations from the last point the state was persisted \cite{rdd_fault_tolerance}.

Finally however, Apache Spark suffers the same deficit as MapReduce and is performed on finite datasets. Therefore streams need to be divided in batches in order to perform calculations. Fortunately, such a library exists :Apache Spark Streaming \cite{web:spark_streaming}. It batches input from streams on regular intervals and supplies it to a Spark RDD environment. The time windows can be as small as a millisecond, therefore it is not formally real time, but can achieve near real-time stream processing \cite{dstreams}.


\subsection{Solution decisions}
\label{sec:solution_decision}
Apache Storm was chosen for a distributed component environment and messaging system. The reason for this was primarily that Storm was conceived with this type of real-time streaming micro-component application in mind. The spouts and bolts provide the perfect building blocks to design an iterative information refinement application with separation of concerns in mind, while the built-in streaming mechanism provides for the distribution needs. However, the lack of exposure for third party integration and the tedious process of specifying each and every component connection will have to be accounted for.

Though Storm contains the means for large scale snapshot aggregation, it will not be employed for it. Instead, the data aggregation will be supported by Apache Spark Streaming. The reason for this is that studies have shown Apache Spark to be upto 5 times faster than both MapReduce \cite{mapreduce_vs_spark} and Storm \cite{spark_vs_storm}. Spark does however have a larger latency, due to collecting batches of data instead of processing them real-time. This however should not cause a significant problem since the envisioned use case is for timed analysis jobs on very large amounts of input data, in order to detect collective tendencies of the system under investigation. For this scope of application the latency issues of Apache Spark do not impose a large deficiency.

Apache Kafka will be employed to facilitate external communication of the platform. The reason for this is its speed and greater scalability. Additionally, but to a a smaller degree, this was chosen because of Kafka's ability to multicast messages. This will allow multiple auxiliary processes to listen in on the proceedings of the platform. With the decision for Kafka comes another benefit, as the Spark Streaming library contains adapters for Kafka allowing direct connection to it. Therefore, data can simply be emitted to a Kafka topic and consumed by a Spark Streaming process. The greatest deficiency of Kafka, being the lack of topic-level order guarantee, is not of grave importance. The hindrance can be overcome by including timestamps or sequence numbers in the passed messages. Moreover, the Spark calculations most likely will not require order retention. The reason for this is that most computations will contain of a \emph{reduce} step, which requires the reduction operation to be both associative and commutative. Therefore the message order disregarded.

\section{Design of the software platform}
These technologies will be adopted by composing them using adapters and abstracting the solutions. The internal implementation details are shielded by abstracting the technologies, simplifying implementation by the user. Some scaffolds for bolts will be provided, intended for different types of data flows and data reductions. Additionally, these technologies are very abstract since they were intended for many unspecified usages. However the (to be developed) platform and group of target applications feature some known commonalities, which were previously considered variations. Therefore some functions can be implemented which were originally intentionally left unspecified. This will reduce the implementation effort required, again simplifying usage of the platform.

\subsection{Micro-component architecture}
The remainder of this section will explain what adaptations to the previously discussed technologies have been made.

\subsubsection*{Apache Storm}
The bulk of the processor (micro-component) construction, execution and messaging tasks of the platform will be performed by Apache Storm. However, as mentioned before, the process of specifying a processor topology in Storm is a cumbersome process due to the necessity of interconnecting each and every process individually. Therefore, cross-connecting $M$ producer components with $N$ consumers requires $M\cdot N$ explicitly specified connections. This is contrasted by technologies that employ topic based channels in which $M$ producers write to a channel to which $N$ consumers are subscribed, requiring but $M+N$ connections to be specified. To this end, a topology builder was developed which enables topic based streaming. The builder will automatically connect the specified components according to the topics they are subscribed to. In this manner a component and its connections can be specified with but a few instructions, as demonstrated in listing \ref{list:topologybuilder}. Note that the complexity of the topology does not impact the amount of code needed, as the code complexity is solely depended on the number of components and not how they are interconnected.

\begin{small}
\vspace{8px}\hrule
\begin{lstlisting}[
language=java, 
caption={Declaration of a processor and communication channels}, 
label={list:topologybuilder}, 
escapeinside={(*}{*)}, 
captionpos=b,
numbers=left,
tabsize=4
]
topologyBuilder.declareBolt(new UserDefinedProcessor("pname"))
	.subscribeAsConsumer("sensor_input_channel")
	.declareAsProducer("debug_channel", "output_channel");(*\vspace{5px}\hrule*)
\end{lstlisting}
\end{small}


Since Storm allows processes to be duplicated for load-balancing purposes, it employs some methods of controlling which duplicated process worker will consume which snapshot. The two chief methods are supported by the platform. The first method is the \emph{shuffle grouping}. It is the simplest channel specification and does not offer any guarantees on which process worker will consume the snapshot. It is therefore described as receiver-agnostic. However this lack of guarantee will not effect most tasks since most will be stateless data processors. The second supported stream manipulation method is the \emph{field grouping}. It is used for processors that do retain a state or somehow require similar snapshots to always be processed by the exact same worker. An simple example of this is a processor that counts the number of snapshots received for each sensor in a WSN. If it cannot be guaranteed that all snapshots of a sensor \emph{S} are always processed by the same worker \emph{W}, one worker might count 40 snapshots and another would count 60 of them. This requires another singular processor that accumulates those counts in order to derive an accurate snapshot count. Therefore it is possible to specify a set of fields which will consistently determine which worker will consume a snapshot. In the developed platform this is specified at topic level. Again, to prevent repeated declarations. Therefore, each snapshot emitted to such a channel is required to include all fields specified in the field grouping of that channel.

Finally, though the abstractions and encapsulations of the Storm platform are believed to simplify implementation efforts, it could still be useful to an implementer to inject their own native Storm bolts or spouts. This might be due to reusing earlier defined bolts or requiring more control of a process than the abstraction offers. To this end, the developed topology builder encapsulates the topology builder provided by the Storm Java library. As a consequence, the topology builder, upon calling the \emph{build()} function, will return an instance of \emph{org.apache.storm.topology.TopologyBuilder}. This allows last-minute injection of self-specified native storm processes, before ultimately generating the Storm topology with that builder.

\subsubsection*{Incorporation of Apache Spark Streaming}
\label{sec:incorporation_spark}
As identified in by requirement \ref{r:basis_accumulated} there is a need to condense the information of enormous amounts of (individually) low-information snapshots into a diminished number of high-information snapshots. Additionally, the large amount of input snapshots and the assertion that the platform should be scalable (requirement \ref{r:scale}) entails that a scalable data accumulator should be made available. 

As specified in section \ref{sec:solution_decision} Apache Spark Streaming was chosen for this task. However, this causes an earlier identified problem: a direct incorporation of Apache Spark in Apache Storm is difficult. In order to solve this inoperability of interfaces it was decided to device a process that functions as an adapter between Storm and Spark. This adapter employs Apache Kafka, for which Spark does provide interfaces, to pipe snapshots obtained from Storm channels. Snapshots are then read from a Kafka channel and batches of snapshots are fed to Spark RDD computations. Once the cloud computations have concluded the data is returned to the Storm environment and aggregated snapshots are eventually forwarded to consecutive processes. This is achieved by deploying two Storm components. Firstly, a specialized Storm bolt named \emph{KafkaEmitter} is deployed. this process simply consumes Storm messages and forwards them to a Kafka channel. Secondly, a Storm spout is deployed which acts as a Spark driver program. This bolt contains the instructions for the distributed computation of the Spark cloud and results of the cloud computations will be returned to it. A graphical representation of this process is depicted in Figure \ref{fig:distributed_accumulator}.

\begin{figure}
\centering
\includegraphics[width=.7\textwidth]{resources/img/distributed_accumulator.png}
\caption{Graphical depiction of the distributed accumulator process}
\label{fig:distributed_accumulator}
\end{figure}

Two interesting remarks should be made, as apparent from Figure \ref{fig:distributed_accumulator}. Firstly, The KafkaEmmitter can be replicated in order to prevent it being a point of congestion in the topology. Secondly, the fact that two distinct components (KafkaEmitter and Spark Master) are present is encapsulated by the topology builder. Developers need only declare an implementation of the distributed accumulator processor (acting as Spark master node) with the appropriate Storm and Kafka channels. The builder will then deploy a KafkaEmitter (or several) and the accumulator. This simplifies deploying the processor and obscures the internals by appearing as a single component.

\subsection{Scaffolds for micro-services}
With the technologies established the the component scaffolds that are supplied to application developers by the platform will be described. First, the base functions shared by all components will be described, before discussing them more in depth individually.

\subsubsection*{Common functionality}
Firstly, the components contain all functionality and information required to emit snapshots to subsequent components. A developer need only package the information in a snapshot consisting of key-value pairs and specify to which stream a snapshot must be emitted. The component then uses the information it received during the building of the topology to route the snapshot to all receivers subscribed to receive it. This not only implies routing the snapshot towards the correct component but also the correct component worker according to the defined field grouping.

Secondly, all components contain a base implementation of the \emph{prepare()} method. This method can be implemented to instantiate some properties that cannot be instantiated in the object's constructor. The reason that some properties cannot be instantiated in the constructor is that Storm processes (stouts and bolts) adhere to a pre-specified execution order. The component is:
\begin{enumerate}
\nospace
\item created by one of its constructors,
\item transmitted to one of the worker nodes of the Storm cluster,
\item further instantiated using the \emph{prepare()} method, and
\item executed according to its specification.
\end{enumerate}
The reason for this course of action is that step 1 is performed on the Storm master node, before distributing the functional object over the cluster. Therefore, during step 2 the object and its members need to be serializable. Non-serializable members are consequently instantiated during step 3, after the object has been transferred and before functional execution. The \emph{prepare(args)} method thus can be used to instantiate certain non-serializable properties. 

\subsubsection*{Spout}
This process mirrors the Apache Storm spout and is the component that introduces snapshots to the network. This component typically contains a handle to some external data source such as a database, API or auxiliary streaming technology. The reason for such a specific processor for this is the special execution cycle it has compared to a Storm bolt. Bolts execute with interrupts. They halt their execution until a new snapshot is available. However, a spout runs on an infinite-loop (until termination) continuously calling a method \emph{nextTuple()}. This method polls, retrieves and emits snapshot depending on the origin of the source.

\subsubsection*{SingleMessageProcessor}
This component is the most basic scaffold and closely resembles a Storm bolt. It however contains some additional functionality that improve its usability. It receives a snapshot and performs computations or analyses on it, before emitting new, enriched snapshots. Its typical use is for transformations of individual snapshots. As noted before this component requires implementation of a singular method: \emph{runForMessage(Message\ m)\emph} which will be called for each snapshot received by the component.

\subsubsection*{HistoricBufferedProcessor}
The HistoricBufferedProcessor resembles the SingleMessageProcessor in that it consumes single snapshots, but instead it processes or analyses an ordered series of relevant snapshots, called a \emph{window}. This is performed by retaining an in-memory buffer to which new snapshots are amended and is periodically filtered on relevance. This component can for example be used to determine recent trends in system parameters. The methods that require implementation for this component are \emph{runForBuffer(List\textless Message\textgreater\ l)}, which is run every time the buffer is updated, and \emph{cleanBuffer(List\textless Message\textgreater\ l)} which implements how and which elements should be pruned from the buffer, should they lose their relevance.

\subsubsection{DatabaseBufferedProcessor}
From a processing perspective the DatabaseBufferedProcessor is similar to the regular HistoricBufferedProcessor. It analyses a buffer of snapshots in order to emit a snapshot containing accumulated or averaged knowledge based on its input snapshots. However, rather then keeping an in-memory buffer of snapshots it maintains a connection with an database. This allows for buffered processing of snapshots that is not performed regularly, thereby not superfluously occupying memory resources.

To keep the component applicable to many database implementations and query languages it was chosen not to instil a database connection. Instead a developer is offered scaffolds to stepwise implement the intended behaviour with an actual  database connection. This scaffolding contains the methods processing the buffer (\emph{runForBuffer(List\textless Message\textgreater\ l)}) and purging the buffer (\emph{cleanBuffer(List\textless Message\textgreater\ l)}) as included in the HistoricBufferedProcessor. Aside from those functions it specifies function end-points for storing a new snapshot into the database and for fetching the relevant buffer from the database, respectively named \emph{persistMessage(Message m)} and \emph{fetchBuffer(Message m)}.	

\subsubsection{DistributedAccumulatorProcessor}
This component aggregates large amounts of laterally relevant snapshots. Laterally relevant entails that the snapshots describe similar data-points, but have little sequential relevance. The input for this process is a large amount of (individually) low-information snapshots. Conversely, the goal of the processor is to emit some high-information snapshot. An example of its usage is combining thousands of snapshots from sensors in order to obtain some collective application-level performance parameters. To accomplish the aggregation of these enormous amounts of data the accumulator principle described in section \ref{sec:incorporation_spark} is employed. By means of the method \emph{runForRange(JavaRDD\textless Message\textgreater\ rdd)} this component offers implementers a reference to the Spark RDD which contains all the snapshots collected during a pre-specified time period. The implementer can then use this RDD reference to sequentially manipulate and aggregate the collection of snapshots. Keeping proper parallelization in mind, this distributed component can perform data enrichment tasks on enormous batches of streaming data.

%A final remark to be made is on the granularity of the batch processing. Some real-time properties are lost by collection and processing streaming data as batches. This has been partly mitigated by employing the windowing mechanism of Apache Spark Streaming. This mechanism collects data in relatively small sub-RDDs. one or more of these smaller consecutive RDD's are then collected as one larger RDD called the 'window'. This window has a fixed size and slides over the sequence of sub-RDDs. This allows these small batches to be part of several consecutive windows. A graphical representation of this process is depicted in Figure \ref{fig:spark_window}. By this method it allows for example the analysis of data windows of the past 5 seconds, every one second. Whereas without this mechanism it would only be possible to process the last 5 seconds every 5 seconds or the last second every 1 second. Additionally, this process is very efficient, since the internal windowing mechanism automatically caches the results of the intermediary sub-RDD's. Therefore the entire chain of computations does not need to be recalculated for each windowed operation, only the transformations past the caching of the sub-results.

%\begin{figure}
%\centering
%\includegraphics[scale=0.55]{resources/img/spark-window.png}
%\caption{Apache Spark windowing mechanism. Source: \cite{spark_user_guide}}
%\label{fig:spark_window}
%\end{figure}

\subsubsection*{AccumulatorProcessor}
This component closely resembles the function of the above described DistributedAccumulatorProcessor, but is executed locally rather than on a cloud cluster. The purpose of this processor is tasks that would otherwise require the distributed accumulator, but whose limited scope be run in-memory on a single worker node. This could be a viable solution for applications that either run the accumulator task often enough or do not collect excessive amounts of snapshots. For these class of applications a locally executed accumulator task should prove sufficient and inclusion of such a components eliminates the base requirement of a Apache Spark cluster to be deployed in order for the platform to be executed, since the DistributedAccumulatorProcessor is the only component that employs it. It should however be noted that not deploying an accumulator in distributed mode could introduce a bottleneck in a Storm topology since the accumulator cannot be duplicated or load-balanced.

The processor was modelled after the MapReduce paradigm \cite{mapreduce} to guide its implementation. An implementer need only specify a \emph{map}, \emph{reduce} and \emph{collect} step.  The exact methods to implement for this are: 
\begin{description}[font=\normalfont]
\item[\emph{map(Message m) : String}] \hfill \\ Computes the key for a key-value snapshots.
\item[\emph{reduce(String key, List\textless Message\textgreater\ l) : Message}] \hfill \\ Reduces sets of key-value pairs grouped by key determined in the map step.
\item[\emph{collect(Map\textless String,Message\textgreater\ m) : Map\textless String,Message\textgreater}] \hfill \\ Collects the key-message pairs emitted by a reduce step. The return value of this method is a map of snapshots indexed by the Storm topic on which it should be forwarded.
\end{description}
Please note that the result of the reduce step is a set of snapshots. It is therefore possible to chain multiple map-reduce steps sequentially, as long as the sequence is concluded with a single collect step.

%TODO beter beschrijven zonder c4 te hoeven lezen
\subsubsection*{ResourceDistributionModelProcessor}
The final component is the ResourceDistributionModelProcessor. This processor is a special instantiation of the SingleMessageProcessor that analyses inbound snapshots according to a pre-specified ResourceDistributionModel (RDM). This model will be discussed in detail in Chapter \ref{ch:rdm}. In contrast to all other processors, this processor is not just a scaffold. Instead it executes completely automatically, requiring only an instantiation of an RDM and a specification of which model variables to output on which Storm channels. The processor then automatically provisions the input variables of the model, calculates the derived values and outputs the requested values as specified.

\section{Demonstration by example topology}
\label{sec:example_application_topology}

\begin{sidewaysfigure}
\centering
\includegraphics[width=\textwidth]{resources/img/example_topology.png}
\caption{Example topology of a platform implementation according to the example case}
\label{fig:example_topology}
\end{sidewaysfigure}

This section will demonstrate an example of a composition of the specified components. For this purpose case exemplified in section \ref{sec:example_case} will be considered. A graphical representation of the topology for the example implementation is depicted in figure \ref{fig:example_topology}. 

As the figure makes apparent, the application encompasses a large number of sensor devices. These devices regularly send their status information to the monitoring application via some external communication technology (e.g. Apache Kafka). These snapshots are introduced into the topology by \emph{SensorSpouts}. These spouts have been duplicated in order to accommodate the large amount of sensors which might send a sudden burst of data. The snapshots are then forwarded to the \emph{SensorProcessors} which have been provisioned with a Resource Distribution Model. This model consumes the measured parameters of the input snapshot and uses them to further calculate all the parameters which can be derived from the inputs, according to the specified model. This model also determines the optimal Resource Utilization Model (RUM) for this sensor device. Should no valid model composition be found this is reported to the \emph{NoRumActuator} which forwards a log message to the \emph{Reporter} component. The \emph{Reporter} will delegate the message to the correct reporting/alerting mechanism, outside of the topology. 

Should the current mode of operation be determined not to be optimal, the \emph{SensorProcessor} will report to the \emph{ChangeRumActuator}. The \emph{ChangeRumActuator} will report requests for change to an entity outside of the topology of the application. The actuator has been implemented as a DatabaseHistoricProcessor. The reason for this is that it will recollect the last few snapshots it received for this sensor and will only actually change the mode of operation of the sensor if it is consistent with the last few snapshots received. This eliminates superfluous communication with the sensor device caused by sporadic behaviour. Alternatively this component could have been implemented as a BufferedHistoricProcessor. However, a sensor is expected to send monitoring data only a few times per day and changes of operation would occur even less. It would therefore make little sense to keep a buffer of the last snapshots sent for each and every sensor in-memory. Additionally, this would have required a field grouping in case the component were to be load-balanced in order to enforce that the request for change of a particular sensor always be sent to the correct worker.

A final transformation to be performed is to infer application level intelligence from the low level sensor statuses. This is performed by the ApplicationAccumulator which collects data for a certain time period and calculates some high level data points, such as the measurement rate of the application averaged over its sensors, the total throughput and how many devices are performing on which RDM. This information is forwarded to the \emph{Reporter} which will make it available for visualization performed outside of the topology. Additionally the accumulator sends its aggregated snapshot to a \emph{TendencyAnalyser} which keeps a sequence of the total bandwidth used during previous time windows. Should this total consistently rise over a period of time, an alert will be sent by the reporter, as specified by the alerting requirements listed in section \ref{sec:example_case}.
	
\section{Discussion of the proposed software platform}
This section will evaluate the design of the monitoring platform. 

\subsubsection*{Satisfaction of requirements}
The first order of business is whether the proposed design satisfies the earlier stated requirements. The message-passing micro-service architecture provides the basis for snapshot transferral and transformation as stated in requirement \ref{r:snaptshot_transformation}. Furthermore, the requirements \ref{r:basis_single}, \ref{r:basis_historic} and \ref{r:basis_accumulated} are satisfied by the inclusion of the \emph{SingleMessageProcessor}, \emph{BufferedProcessors} and \emph{AccumulatorProcessors}, respectively. Finally, the last two requirements regarding the size of the applications in the problem domain and entailing scalability of the solution have been decisive for certain choices of the supporting technologies. For example, they it is reflected in the employment of cloud processing technology Apache Spark. From the aforementioned arguments it is concluded that every requirement is represented and met in the design of the platform.

\subsubsection*{Completeness with respect to QoI attributes}
The goal of the platform is to process and enrich data. It is therefore rational to evaluate the appropriateness and completeness of the platform by considering the information processing capabilities it offers. This section thusly evaluates the platforms completeness by demonstrating that the platform only positively impact the Quality of Information (QoI) of the input data. This entails that the QoI is improved or retained, but never lost as data passes through a platforms topology. This will be achieved by arguing the QoI parameters which were enumerated in section \ref{sec:back:qoi}. 

The first consideration of QoI is regarding the processing of data by the platform and affects the precision, completeness and usability of information. Firstly, \emph{precision} and \emph{certainty} are obtained by employing the HistoricProcessors. By averaging measurements anomalies are mitigated and the measured value closely approaches the norm of the measurements. Provided that the accuracy of the measurements is sufficient, this improved precision should consistently yield a measurement near the actual value. Secondly, the \emph{Usability} of information is improved as data moves throughout the topology. To illustrate this a thought experiment is proposed, using the example topology listed and described in section \ref{sec:example_application_topology} and a batch of raw data emitted during a certain time window. Before the data enters the platform it contains the potential to calculate the average throughput offered by the sensor application during that time window. Instead, this process is automated by an implementation of the platform and the resulting information is offered for further processing or visualization. This demonstrates that the platform can facilitate usability for information by calculating and producing ready-for-use values. It should however be noted that the \emph{completeness} of the information is greatly reduced during this process. To illustrate, from the average application throughput the throughput for individual devices can no longer be determined. For this reason, and others which will become apparent, committing the raw data to storage before processing it is recommended.

The second class of QoI attributes regards the processing efforts, expressed in time and costs. As the relevance of information degrades as time progresses, timely processing is paramount. \emph{Timely} execution is achieved by providing a scalable distributed solution. This ensures that, regardless of the intense information \emph{throughput}, the calculations can be performed in near real-time. Notice that only \underline{near} real-time is claimed, since Apache Spark collects records during a time window and performs calculations in batches. However the time window of such a batch can be set arbitrarily small for fine grained processing. Thereby it does not impact the timeliness significantly. However, adverse to this gained timeliness is a decreased \emph{affordability}. In order to incorporate these distributed cloud technologies a cluster of machines and increased development resources will need allocation. When the solution does not require this degree of scalability this poses an undue burden. Therefore locally deployable alternatives to these distributed processors are also provided. Implementations of the platform are therefore offered a trade-off between timeliness and cost.
	
Lastly, are the \emph{tunability} and \emph{reusability} of the information. Firstly, the data can be duplicated among different communication channels which allows differentiating calculations to be performed on the same data. Secondly, in order to facilitate evolution of end-user demands the platform has been designed with separation of concerns in mind. This allows continuous reconfiguration of the platform to be performed with reduced occurrence of concern entanglement. By redeploying the topology the same raw information can be used to facilitate updated user demands. This is also another reason to store the raw data before processing it. 
%By caching the data it can be re-fed into an updated topology in order to initialize an application as if it had been running for days.

Some final remarks should be made on the analysis. Firstly, the platform cannot offer any improvement or retention of information \emph{accuracy}, as it is solely determined by the method and quality of data measurement. Secondly, it should be noted that the platform does not assure preservation of any of these claim, since an implementation of the platform can violate any guarantee made. It can only be claimed that the platform does not impede any of the parameters and offers the means for developers to develop applications that do guarantee it.

\subsubsection*{Ease of adaptation}
A second point of focus is the ease of adoptation provided by the platform itself. It is asserted that low-level implementation details of Apache Storm and Spark are effectively obscured. This was achieved by offering some abstract components that require implementation of only a few methods. This obscuration entails a clearer programming interface to an implementer, as stated by the \emph{facade} software design pattern \cite{facade_pattern}. 

Secondly, the provided topology builder facilitates easy and fast building of a Storm topology. It does so by providing context aware topology and process instantiation, and topic based communication subscription and emission.  As mentioned before this allows $M$ producers and $N$ consumers connected by a single topic to be connected with complexity $\Theta(M+N)$, instead of the complexity $\Theta(M\cdot N)$ which would be required without the concept of topics.

These assertions will be formally validated in Chapter \ref{ch:validation}.

\subsubsection*{Technology stack}
Another issue to contemplate is the technology stack required for the platform. As mentioned in section \ref{sec:solution_decision}, Apache Storm was chosen as chief enabling technology. The main reason for this is that it offered most of the features required and would reduce the technology stack. However, by employing Apache Spark for distributed data aggregation, two additional cloud technologies are introduced. Spark itself and Kafka which is required in order to be connected to a Storm Topology. However, the inclusion of a distributed aggregation is necessary in order to keep the computations scalable. Additionally, the speed and efficiency arguments raised in section \ref{sec:solution_decision} justify the deployment of these additional technologies. Finally, when this scalability is not required Apache Spark and Kafka clusters can be executed locally on a single machine. This would still enjoy benefits from process parallelization, without requiring cluster deployment. Finally, Spark and Kafka may be omitted entirely, if permitted by the snapshot influx, as a non-distributed accumulator is also included in the platform.

%TODO \/ weg? ook al in validation en conclusion
\subsubsection*{Future work}
Finally, the topology-based separation of concern approach allows for visualization of the computations and distribution. The chain of computations can easily be depicted as a directed graph with processors and topics as nodes and processor-topic connections as vertices. Such a topology visualization would for example be very useful for identifying incorrectly or disconnected components. While runtime analysis tools are available for Storm topologies a graphical modelling/development tool is lacking. Such a tool would allow a topology to be drawn and functional methods to be implemented later. Though not featured, future efforts could be made to facilitate them.

%ease of use
%large technology stack
%qoi metrics
	%datastreams


%ease of use
%	topology build by 2 lines of code per component (4 if formatted)
%		create, register, declare, subscribe
%	total for test topology = xxx
	
%large stack -> also supply old accumulator
%also supply old
%		lower stack
%		gebruikt during development
%		map-reduce
%			rdd not map-reduce
%		Spark can run locally
%???	
%1. Every stream ends in distinct collection/number of streams -> therfore no n*m analysis necessary (only for each stream and for each analysis individually. (htis only descirbes outputs, not the inputs)
%2. there are no one/N to many relations. Implications?
%+RUM

%no mass emitter?
%??


%Discussion aan de hand van qoi metrics

%future work?
%composer GUI?



%architecture
%	model reification
%what components needed
%	featuremodel
%	requirements
%which candidates
%benchmarking
%desicions

\newcommand{\rumid}{1}
\chapter{Resource Distribution Model}
\section{Requirements}
In this section we will [discover] the requirements for the [method of] the RDM. We will achieve this by performing an commonality/variability analysis [ref]. This will [tell] us what the common [features] are which we may [assume] and the variation for which we will need to [design].
\subsection{Commonality/variablity analysis}
\subsubsection{Definitions}
\begin{description}
\item[Resource:] Any measurable/calculable parameter of a system
\item[Component:] Any physical or hypothetical entity that can consume or produce a resource
\item[QoS:] ...
\end{description}
\subsubsection{Commonalities}
\begin{enumerate}[label=C\rumid .\arabic*]
\item Any resource can be consumed or offered by a component
\item A resource can be produced or consumed by multiple components
\item Resources are scarce. I.e. the amount produced must exceed the amount consumed.
\item \label{c:res_transf} Resources are correlated and can converted into one another.
\end{enumerate}

\subsubsection{Variabilities}
\begin{enumerate}[label=V\rumid .\arabic*]
\item \label{v:obvious} Obviously, we cannot predict all resources, constraints and components that might be used.
\item \label{v:micr_macro} Resources of a system can be modelled on a micro-scale or macro-scale.
\begin{itemize}
\item A micro scale (e.g. a single sensor) entails concrete, palpable parameters.
\item A macro-scale (e.g. an entire WSN application) entails accumulated, theoretical parameters
\end{itemize}
\item \label{v:nr_optimizer} A system can have multiple resources as QoS indicators
\item \label{v:granularity} short term resource usage (e.g. seconds) requires a different granularity than long term resource usage (e.g. interval in days).
\item \label{v:measure_vs_derive} Some resources are directly measurable and thus [fixed] for a certain moment of measurement. However, some resources are derived and calculated using other resource values. \cite{feature_model}
\item \label{v:state} Most resource values differ depending on system's measured state
\item \label{v:function} Some resource values differ depending on a specific system function
\end{enumerate}

\subsection{Requirements}
\begin{enumerate}[label=R\rumid .\arabic*]
\item \label{r:main} The model should represent resource distribution in a system
\item \label{r:transform} Resources should be able to be transformed into other resources (many-to-many)
\item \label{r:resource_types} The model should account for the fact that the value of a resource can originate from different sources. These sources are the following (accompanied with an example):
\begin{description}
\item[constant] a predefined value specified on development time (e.g. initial battery capacity),
\item[measured] a value specified as observed on evaluation time (e.g. percentage of battery capacity left),
\item[calculated] derived from measured values (e.g. runtime left),
\item[variable] any value or a calculation depending on specific system function (e.g. power usage).
\end{description}
\item \label{r:optimizer} Each model should have one, and only one, resource that is associated with a heuristic optimizer function.
\item \label{r:calculable}Given a resource distribution model, constant-valued resources and measurements, for each combination of values for variable resources, a value should be able to be evaluated for each calculated resource
\item \label{r:solvable} Given a calculable  resource distribution model (\ref{r:calculable}), a set of resource constraints and an optimizer function; an optimal, valid appointment for each variable resource value should be able to be solved efficiently.
\end{enumerate}

\section{State-of-the-art}
[Some] work regarding modelling resource [distribution has been performed in several studies. An elementary example of such research is the studies of [name][ref]. Through their efforts they [layed] the ground work for representing entities interconnected by shared resource. This UML-based model was one of the first examples of such a representation using formal [principals]. Another example of early research is the study performed by [name][ref]. This study focussed on modelling resource utilization in embedded systems using timed state machines. The transitions in these automata were attributed resource costs to model the consumption of resources for remaining in a state or transitioning to another. [Possible] resource consumption and performance could then be calculated and analysed. 
A continuation of this work was performed by [lalala] et al [ref]. They [combined] the approaches of the previous authors by provisioning the modelled software/hardware components each with their own state machine. These state machines model the resources and services that are offered and required by the components. By [extracting/analysing] these component models as composite state machines, model checking tools (such as UPAAL [ref], [more?]) can be used to analyse and evaluate the performance of the system under investigation.
%TODO voorbeelden hebben maar 1 RUM

\section{Solution}
\subsection{Solution options}
These efforts have produced suitable methods of representing components connected by shared resources. Espesially the notation of [name 3e][ref] which is both intuitive and [descriptive]. We will therefore continue to use this notation throughout this [paper]. 
however these models are all focussed on components that are self-aware of their resource usage and performance. Instead, we are interested in [off-site] analysis of interconnected resources and accumulated performance of the entire system. Our focus is therefore alternatively more resource-centered. It is concerened how production and consumption of a resource is interconnected, with components serving as secondary [elements] merely specifying how these resources are converted to other resources. Therefore a resoucre-centered adaptation of this framework might be more suitable for our problem.
%TODO meer

Secondly, there is the issue of how to [represent] the RUM, the model for variable behaviour of components. Previous [attempts] [refs] have used timed automata to represent behaviour cycles. This allows for automated tools to calculate a runtime schedule in [incredable] levels of granulrity. However the high level of granularity comes at the cost of efficiency. When we shorten the [time] interval the [system] requires additional time and computational resources. This might [impose] a problem on resource constraint devices or applications that require the [program] to run many times for [many] [devices/systems]. Additionally, we need to consider that a component might be able to operate according to differing RUM'S for which a valid, optimal RUM needs to be determined. In the worst [case] these RUM's [influence] each other, which implies that for each composition of models the individual models need to be re-evaluated [%TODO validate]. 

An alternative approach is to model the RUM as a flat set of paramters. This is achieved by averaging the behavour otherwise modeld by timed automata. This comes at great cost of granularity, since the RUM's now only describe static, long-term behaviour. However it significantly improves the complexity of the search space, since the only exponential factors are the number of variable components and the number of RUM's for those components. For this approach timed automata is no longer a sensible technology since the element of time intervals has been eleminated. Instead the problem is pure decision [beter uitleggen, ref] which's search space can be explored with a simple brute force search. However more effectively, combinatorial problems can often be solved with constraint solvers. The problem is easily [transposed] to a constraint problem with the resources as model, resource constraints as constraints and the RUM's as variables for the variable components. With the many solution strategies described in \ref{subsec:constraint} available for different types of problems, a suitable solver must be [possible].

\subsection{Solution choices}
With careful consideration the following choices for the solution implementation have been. For modelling we chose to [adapt] the framework of [ref] by emphesizing on resources. This allows constraints to be [added] to resources, modelling the limits and requirements of the system. The components will still exist in the model, but will merely serve the function of connecting two ressources to one another. Another adaptation is the existence of multiple RUM's for a component, which allows calculation of the optimal system functionality.

As for who to model the RUM, we chose to reduce the complexity of the system by modelling variable resource usage with static parameters. The strongest advocate for this choice is the fact of the focus for this research: large IoT applications. In an IoT monitoring platform the RUM [determination] process will need to be performed repeatedly for many sensor devices. Additionally, most IoT [refs!!!] devices only send and recieve data a few times per day. Therefore high granularity is not of grave importance because the feedback-control cycle is not that short. However, eventhough we do not use timed automata at runtime, they are still a valueble technology to be used when developing and testing the static parameters a RUM at develop-time. The choice for static RUM's implies constraint programming as a suitable model solver paradigm. 

\section{Design}
\subsection{Model}
A graphic representation of the adapted metamodel can be found in figure \ref{fig:component}. To illustrate the application of this metamodel, an example of an instantiation of the model can be found in \ref{fig:rdm_cpu_radio}. In [essence] the model is a collection of \textbf{Resources} and \textbf{Components}. Each of these resources can be connected to a component by means of a \textbf{ResourceInterface} and a \textbf{ResourceFunction}. 
\begin{figure}
\centering
  \includegraphics[width=0.5\linewidth]{resources/img/component.pdf}
  \caption{Notation of an RDM component with RUM's}
  \label{fig:component}
\end{figure}
\begin{figure}
\begingroup\centering
  \includegraphics[width=\linewidth]{resources/img/rdm_cpu_radio.pdf}\endgroup \\ \\
  \noindent Constraints: \\
$c_1: cycles_{clock} >= cycles_{CPU}$ \\
$c_2: power_{power\_source} >= power_{CPU}+power_{Radio} $ \\ \\
\noindent Optimize:\\$max(QoS)$
\caption{Example instatiation of the RDM meta-model with a CPU and a radio}
  \label{fig:rdm_cpu_radio}
\end{figure}

%TODO insert pics
\subsubsection{Resource}
A resource is an entity describing a parameter of a system. This can be a measured parameter (e.g. battery capacity left or throughput), but can also describe a derived parameter (e.g. service time left). Each resource is identified by it's name and has a unit associated with it. By aggegating the ResourceInterfaces of a resource the amount of the resource produced and consumed can be collected and analysed.

\subsubsection{Component}
Any entity producing, consuming and converting a resource is represented by a component. A component can therefore be a physical entity such as a radio module or a battery or a hypothetical entity such as a QoS calculator executing a heuristic function. A component [contains] a ResourceFunction of each Resource it is connected to.
A [special] case of the Component is the ModelComponent. This class inherits all functionality of the ordinary Component, but its ResourceFunctions are extracted from one of its RUM's. Each RUM describes the parameters during one mode of operation of the components. This allows runtime analysis of variable behaviour as effect of different functionalities.

\subsubsection{ResourceInterface}
Resources and components are connected through resource interfaces. A ResourceInterface can be one of three types:
\begin{description}
\item[Offer] Indicating that the component produces an amount of the resource,
\item[Consume] Indicating that the component consumes an amount of the resource,
\item[Calculate] Special consume relation. This connection supplies 100\% of the offered resource, without formally consuming any amount. This relation is used to further calculate with the offered value, without it impacting the constraints of the resource. For example a QoS indicator that is ``consumed'' by a general QoS calculation.
\end{description}
Each interface has a value specifying the amount of the resource produced or consumed by the component. This value is repeatedly set and evaluated at runtime by executing a ResourceFunction.

\subsubsection{ResourceFunction}
The value of a ResourceInterface is determined by a ResourceFunction. This function constist of a function that takes a double array as argument and with a double as result, and an array of resource identifiers to fill the input array respectively. ResourseInterfaces can [compactly] be instantiated using lambda expressions and varargs. E.g.:
\begin{lstlisting}[language=java, frame=single, numbers=left, tabsize=4, basicstyle=\small]
ResourceFunction totalServiceTime = new ResourceFunction(
	(x)->x[0]+x[1], yearsServed, yearsLeft
);
\end{lstlisting}

To model the [gewenste] behaviour of the model we introduce a set of \textbf{Requirements} and an \textbf{Optimizer}.
\subsubsection{Requirement}
A resource can have any number of Requirements function as constraints that limit the possible values of [variation] of that resource. The standard built-in requirement for every resource is the \emph{OfferConsumeGTE} requirement which enforces that the amount produced needs to be greater or equal than the amount consumed. Additional requirements \emph{OfferConsumeEQ} and \emph{RangeRequirment} are supplied that respectively require the exact amount offered to be consumed and the amount offered or consumed to be within certain bounds. Finally the abstract class Requirement can be extended by a developer to specify any tailored requirement.
\subsubsection{Optimizer}
To [assertain] the heuristic [grade] of a RDM with a [gevulde] RUM configuration we introduce the Optimizer. The Optimizer is an extended class of Resource of which exaclty one must exist in an RDM. The optimizer takes the evaluated offered amount of the Resource and calculates a score. This score is a value on a comparative scale of which a higher value entails a more optimal solution. Supplied are the \emph{MinMaxOptimizer} which evaluates that the amount offered must have a minimal or maximal (specifiable) value and the \emph{ApproxOptimizer} which evaluates that the resource must have an amount offered as close to a specified value as possible. However, custom implementations of the Optimizer can again be made by developers.

\subsubsection{RdmMessage}
Finally, to supply the model with the state of the system under investigation, we pose the RdmMessage. The RdmMessage is provisioned using values measured from the system and injected in the model, after which the appropriate resource values are evaluated accordingly. Technically, a simple mapping from a resource name to a measured value value would do for this purpose, but this mapping is wrapped in an object to support future [expansions] of the object.

\subsection{Solving the model}
With the model well-esteblished, we can now try and solve the model. From requirement \ref{r:solvable} we find the goal of solving the model is to find a composition of RUM's such that:
\begin{enumerate}
\item each ModelComponent has exaclty one RUM associated with it,
\item all resource constraints are satisfied, and
\item the optimizer function of the optimized resource has the highest value.
\end{enumerate}
The first and second requirement implies constraint solvers as an applicable technology[. Since] they are effective in finding a valid solution for a constraint decision problem. However, the third requirement [implies] that we do not want to find any valid solution, but the optimal valid solution. In order to do that we need to consider \emph{every} valid solution to the problem and compare how they compare [heuristicly]. This entails a [brute force] search approach through the entire search[-]space of RUM compositions. We can however use constraint solver [paradigms] to preventively reduce the search space as we search through it.

The way we do this is by employing backtrack search. In a simple brute force search we would calculate all RUM compositions and for each composition we provision the full model and evaluate it. Instead we will iteratively select a component and one of its models. We will then not provision the entire model, but inject the selected model in the chosen ModelComponent. Consequently we calculate only those variables we can resolve with the information currently represented by the model. We then evaluate the resource constraints. Given an incomplete model any constraint can have one of three statuses:
\begin{itemize}
\item satisfaction,
\item failure, or
\item uncertain
\end{itemize}
for every consequent assingment of unprovisioned components.

If a constraint evaluates to \emph{satisfied} it will be pruned and not [evaluated] in the remainder of this [branch] of the search [tree], since we know it will always succeed. If a constraint is \emph{uncertain} we keep it, since we do not know its [state] for each and every future [state]. If even a single constraint \emph{fails} we know the remainder of this branch of the search tree will never be valid. Therefore we backtrack through the tree by partially rolling back model assignment. We then select a different model for the same component or a different component entirely and repeat the algorithm [ref to algorithm]. This way we do not re-evaluate constraints we already know the state of and do not [visit] paths we know will not satisfy the constraints. Given that we encounter unsatisfactory options early in the tree, this will eliminate large [parts] of the search tree. An example of this algorithm on the example posed in Figure \ref{fig:rdm_cpu_radio} is given in Figure \ref{fig:search_cpu_radio}. This application illustrates that using this algorithm, we eliminate a significant portion of the search tree. This is due to early constraint failure detection in the \emph{CPU=high\_cpu} banch of the tree.
\begin{figure}
%\documentclass[11pt,a4paper]{article}
%\usepackage{qtree}
%\begin{document}
\hrule
\vspace{10px}
\Tree [. {CPU=low\_cpu\\c_2 \hspace{1px} is invalid\\backtrack} [.{CPU=high\_cpu\\c_2 \hspace{1px} is valid\\prune c_2} {Radio=high\_radio\\c1 \vspace{1px} is invalid\\backtrack} [.{Radio=low\_radio\\c_1 \vspace{1px} is valid\\prune c_1} {valid composition found\\calculate optimizer score} ] ] ] \\

\noindent Legend: \\
\emph{Assignment\\Observation\\Action\\}
\hrule
%\end{document}
\label{fig:search_cpu_radio}
\caption{Application of backtrack search on RDM of Figure \ref{fig:rdm_cpu_radio}}
\end{figure}

\section{Discussion/evaluation}
why no state machine (rum\_ basis\_ 2, rum\_ basis\_ 89) To much calculation, repeatedly
	state machines are usefull for developing rpm's
(dis)advantages van explicit model

%\subsection{discussion}
%always possible to convert state to optimizable (hypothetical) heuristic resource [search def:heuristic]
%short vs long term
%	solved by explicitly focussing on long term. By choice, long rtt and long lifetime.
%	allows collapsing states to single state (single variable)
%	guarentees solvability
%measure vs derive
%	- every internal resource needs to be calculable from only eventual external, measurable resources %(transitive)
%	- no cyclical calculations
	





%What to represent
%	Components, calculators
%	resources
%		requirements
%		optimizable
%	resource distribution
%What inputs
%What outputs
%what actions





%\chapter{Design method}
\section{Adaptation}
Application of the Design Cycle to these design artefacts
\section{Cycles architecture}
\section{Cycles RUM}

\newcommand{\idsystems}{\nedap Identification Services }
\newcommand{\nedap}{Nedap }
\newcommand{\ublox}{u-blox }
\newcommand{\sensit}{SENSIT }
\chapter{Proof of concept by case study}
\label{ch:validation}
\section{Case study}
\subsection{Background}
\label{sec:sensit}
\subsubsection*{Nedap - Identification Systems}
[TODO]
\subsubsection*{\sensit [smart parking] application}
The \sensit \idsystems smart parking application is devised to monitor parking lots and garages. It employ a huge amount (up to thousands) of affordable LPWA sensor nodes. Each individual parking spot is equipped with one of these sensors to determine its occupation. To determine changes in occupation, each sensor is equipped with an infra-red and magnetic induction sensor. Should a change in occupation be detected, a message containing the measured sensor deltas is sent to the back-end application. This granular approach to smart parking allows the \sensit application to monitor and visualise the occupation of individual parking spaces in a lot, garage or even across cities.

In order to communicate with the back-end the sensors employ wireless technology. Previously, the sensors were connected to sinks using a proprietary network of relay nodes. However the recent proliforation of large scale cellular IoT networks has caused \nedap to shift towards these technologies. This allows large numbers of sensors to a single cell tower, without the need of deploying and managing a network of relay nodes for new sensor deployments. Additionally the effort in managing and maintianing the network is outsource to professional operators. To connect the sensors to the internet the \emph{Narrow-band Internet of Things} technology was determined to be most suitable. New \sensit sensors are therefore equipped with \ublox \cite{web:ublox} NB-IoT radio modules to connect them to operated cell networks.

\subsection{Context of the Case Study}
In this section we will describe and scope the context of the QoS monitoring application to be developed. We will first describe the input for the application in terms of sensor data emitted by the WSN application under investigation. Consequently, the characteristics of the expected outcomes of the application to be prototyped will be discussed. 

\subsubsection{Sensor data signature}
The sensor devices send a message with key point information (KPI) data along with every data message it sends. Alternatively, it will send one of these messages periodically if no data messages are sent for [time period]. When computed universally, a message rate was determined of about 15 messages per sensor per day. However a specific per sensor analysis yields a message rate of between 10 and 50 KPI information messages on average per day, with some outliers for more active sensors which can reach up to 250 messages per day on a regular basis.

The data sent by the sensor contains some typical networking data points, such as source IP address, source port, source device ID, message sequence number and a timestamp. Additionally the message contains a hexadecimally encoded string describing the KPI's collected by the \ublox radio module. The data collected by the \ublox module contains mostly data points depicting the signalling functions of the radio module. Such KPI's include the signal-to-noise ratio, signal quality (RSSI), Extended Coverage Level (ECL)  and more. Additionally the KPI information includes some physical attributes of the radio module. Attributes such as the module's uptime, number of restarts and temperature. 

%TODO recalc nr byts!!!!
The ordinary data plus the \ublox KPI data are contained within [128] bytes of data (\nicefrac{1}{2} KiB). Considering the messaging rate of a typical sensor we yield an imposed per sensor footprint on bandwidth of 5-25 KiB/day for the majority of sensors, with outliers of 125 KiB/day for extremely active sensors.

At this moment only a few nodes equipped with the NB-IoT technology have been deployed. Therefore a large scale test bed for the to be prototyped monitoring application does not exist. Therefore a simulated sensor environment has been devised to test the prototype application for contemporary and near-future smart parking applications. This simulation is based on data signatures and values observed over a half year period emitted by the few nodes that have been deployed.

\subsubsection{QoS monitoring needs}
In collaboration with \idsystems a list of requirements for the outcomes of the prototype was compiled. These consequences are to be effected by the prototype application, based on input from (simulated) sensors. However, the actual implementation of the prototype is secondary to this chapter, since the primary goal is to evaluate choices made for the underlying development platform. Therefore a comprehensive, formalized requirements document has not been included in this thesis. We will however shortly describe the features required of the monitoring application to be developed in order to contextualize the implementation efforts of the prototype.

The consequences the application must effect are classified into three categories. The first of which is sensor feedback. This entails commands sent to sensors to alter its execution strategy, based on observations made in the monitoring application. This can be based on individual sensor data, historic sensor data or higher level data snapshots (e.g. sink level). An example of such feedbacks are to decrease data rates to guarantee a predetermined minimum sensor lifetime or due to poor cell connectivity. This functionality is currently not present in the \nedap sensors, but is intended in the future. Therefore it will be implemented into the simulation environment to test the command \& control capabilities of the platform.

The second type of effect to be caused by the application is instant alerting. The primary use case for this kind of consequence is when physical maintenance is imminently required in the application or its network. Detectable causes of when this might be warranted have been deliberated with \idsystems and examples include:
\begin{itemize}
\nospace
\item a long term drop in coverage level which might indicate permanent obstruction of signal
\item extremely high temperature readings indicating an electrical malfunction
\item unusually long periods of inactivity or, conversely, extreme data bursts indicate a rouge node not executing according to a valid strategy.
\item calculations estimating node lifetime determining a node needs replacing.
\end{itemize}

The last type of consequence is reporting. The goal of this is to inform technicians, managers or clients on the general operation of the WSN application. This comprises two types of reporting. The first is \emph{periodical reporting}. Periodical reporting will primarily focus on business goals such as long term performance metrics, compliance to service level agreements of both service providers and clients, and prospected short-term maintenance efforts and costs. The other type of reporting is \emph{real-time reporting}. This is useful to technicians monitoring the performance of an application during its runtime. Use cases include monitoring the number of incoming events, latencies of sensor devices and sinks, environmental conditions (such as weather and temperature) and which sensor strategies currently are deployed. Notice that the real-time aspect of this type of reporting does not require events to be reported instantaneously since for such statistics a per second or minute update suffices.

\section{Structure of the validation study}
With the application, case and its context clear, the focus will be turned to detailing the validation study. Before executing our validation study, this section will first depict the taken process. We will begin by clearly stating the claims we aim to confirm and the bounds of our scope. Following that we will describe the intended method of testing those claims specifically by detailing the quantified criteria the platform implementation process must adhere to. We must note that these criteria will only cover the scope of the validation study, not the functional requirements of the implementation for the case. As mentioned before, though important for the outcome of the product for the company, for this validation study these requirements are ancillary.

With the goals clearly stated, parametrized and quantified, we will design and implement a prototype monitoring application built upon the developed software platform, tailored to the QoS monitoring needs of \idsystems. As mentioned before the actual implementation details are secondary for validation purposes of this chapter. Therefore we will only touch upon it shortly without going into great detail. We will however give a short summary of the developed prototype to provide a context to the validation efforts. During and after the development process we will measure the relevant parameters required to evaluate the determined validation criteria. To conclude the investigative implementation, we will attempt to adapt the constructed application to a few hypothetical extension scenarios in order to explore the adaptability of the provided platform.

We will conclude this chapter by stating, analysing and deliberating the results obtained by measuring and observing the development process. These results will be compared with the priorly determined criteria of the study. If these criteria are met, this will validate the claims they are meant to affirm. We will finish by discussing the process and results in order to deliberate the limitations and lessons learned regarding the proposed development platform.
	
\section{Criteria of the Case Study}
\subsection{Claims}
\label{sec:claims}
In this section we will state the claims regarding the proposed platform we aim to validate. The cardinal claim investigated is that the appropriate level of abstraction was chosen in the design of the development platform. This entails that our collection of components can be adapted to suit a plethora of purposes and target applications. Conversely, the level of abstraction is not that low-level that every implementation requires unnecessarily large development efforts because basic procedures require repeated implementation. This claim mirrors the research question \ref{rq:abstraction}, which asks "What is the appropriate level of abstraction for a WSN monitoring platform [...]". This claim is explicated into three sub-claims.

The first sub-claim regarding the level of abstraction is that the platform features a level of abstraction low enough to facilitate the implementation of the monitoring application for \sensit. I.e. the platforms abstraction does not obfuscate key functionalities which would require reimplementation of formerly present features. 

The second sub-claim to be validated is that the level of abstraction is not that low that it requires application developers to repeatedly implement functionality that, due to their frequent nature, should have been provided by the platform itself. This claim seems similar to the first claim. However, as will become apparent in Section \ref{sec:val:method}, the metrics and methods verifying these claims are different. Therefore they shall be treated as two separate claims.

The final sub-claim employed to validate the appropriate level of abstraction is that the platform facilitates convenient adaptation of a realized platform implementation. This validation will be performed by introducing or changing a minor feature. Examples of such features could be new reporting goals, variations to the input or change to the requirement context. Should the appropriate level of abstraction have been chosen, it should prove uncumbersome to adapt the topology to these novel conditions.

To recap, we validate the main claim by three sub-claims that are summarized as applicability, usability and adaptability. For the remainder of this chapter these three claims will be addressed using these headings. In full these claims read:
\begin{description}[style=nextline]
\nospace
\item[Applicability] The platform's level of abstraction is low enough to suit a large number of applications,
\item[Usability] The platform's level of abstraction is high enough that the framework prevents repeated implementation of common procedures, and
\item[Adaptability] The platform facilitates effortless adaptation of an instantiated application.
\end{description}
Altogether, these claims culminate in the main claim that the appropriate level of abstraction has been chosen.

%The first claim regarding our platform is that the appropriate level of abstraction was chosen. This entails an adequate compromise between the ease of implementation of the platform and the flexibility of its components. We claim that we have chosen our level of abstraction in such a manner 

%The second method employed to validate the level of abstraction of the platform is by extending the developed application. This will be performed by introducing a new feature. Examples of such features could be new reporting goals, variations to the input or change to the requirement context. The degree of adaptability will then be measured by the amount of topology components that need to be introduced or changed. Should the appropriate level of abstraction have been chosen, it should prove uncumbersome to adapt the topology to these novel conditions.

%The final claim that requires validation is regarding the scalability of the platform. As mentioned numerous times before the extreme scale of WSN applications requires (auxiliary) back-end processes that are at least as scalable as the application they observe, as is captivated in research question \ref{rq:design_scale}. Our claim is that our platform offers the tools to design a fully scalable WSN monitoring application. In order to validate the scalability of the platform and its implementation, possible congestion points will need to be identified and stress-tested in order to show that proper configuration of the component(s) will alleviate any scalability issue. This will be validated by means of two methods, which will be detailed in section \ref{sec:val:method}.

%TODO move to structure/methodology
\subsection{Bounds}
Before going into how we aim to validate the stated claims, the bounds and limitations of this validation study will need to be considered. The first glaring limitation of this study is that it is extremely limited in scope. The platform will only be implemented for a specific WSN application and this study will therefore not state the platform to be appropriate for the entire set of applications that was determined in Section \ref{sec:back:context} of the background chapter. Instead, this study will at most affirm the platform as a proof-of-concept for WSN application QoS monitoring.

The second limitation worthy of notion is that, aside from only regarding a single WSN application, it will also run on a simulation of that application. As mentioned before, this is because the NB-IoT incorporated sensor devices of the \sensit application have only recently started deployment. As a consequence a test bed of significant scale is presently not available. However by simulating a full future deployment of the application we are able to easily adapt the WSN application under investigation, in terms of both scale and functionality. This allows us to not only test for intended regular behaviour but also for extreme and niche conditions. Additionally our simulated environment allows for easy temporal manipulation, which enables us to speed up, halt and repeat simulations.

%only one case. Partial validation, proof of concept
%simulated environmant (based on acutal data signatures and data rates)
%	current state of sensit not a scale challenge
%	simulation can fast forward
\section{Method}
\label{sec:val:method}
\subsection{General approach} 
In order to validate weather the level of abstraction of the platform can facilitate the needs of the intended monitoring application for \sensit (applicability), a prototype implementation will be designed and constructed. The expected outcome is an instantiation of the platform that serves the QoS processing needs, without requiring to work around the platform and without breaking the abstraction in order to access underlying processes. The possible existence of such an instantiation demonstrates that, at least for this use case, the level of abstraction is low enough to expose the full functionality of the platform, validating the first claim.

In order to validate that the level of abstraction is low enough, but not too low (usability), we will consider the program instructions required for the platform instantiation. These required instructions should not be more then the instruction required for a hypothetical monolithic implementation, supposing the level of abstraction is not too high (applicability claim). 

Finally, the adaptability of the platform and its instantiations will be evaluated by introducing some minor new features and requirements to the platform implementation. Should the appropriate level of abstraction have been chosen, it should prove uncumbersome to adapt the topology to these novel conditions.

From a business perspective, the most interesting parameter to express the adoption effort would be the time required to develop and evolve an application based on the proposed technology. However, this parameter is extremely subjective as it heavily depends on the level of skill of the developer and its familiarity with the technology. We will therefore primarily measure the effort by the code, expressed in number of instructions, required to construct a monitoring application built by integration of our platform. 
%TODO voor presentatie: goldylocks area.

%Our method of confirming the scalability of the platform is twofold. First, we will flood the system with events. If our claim of scalability is correct this will not cause a build-up of message anywhere in the topology of the application. Should such a congestion occur this should be able to be alleviated by scaling the deployment configuration of the components, i.e. the number of tasks and workers per component, without requiring a change in the topology or the internals of the components themselves. The second method we will employ to test the scalability of the system is by initially configuring the simulation in for real-world deployment. We will then deliberately trigger an event shower in the sensor simulation. It is expected that the platform will experience a sudden influx of input messages. Should such an event occur, the platform is not expected to hold its ordinary timing constraints. However, the platform is expected to eventually return to its normal execution, i.e. within the bounds typical of ordinary execution. The platform will pass this test if it is able to process the batch of messages caused by the sudden influx and return to ordinary execution within a certain amount of time.

\subsection{Validation criteria}
\label{sec:criteria}
Before starting the implementation, the criteria the monitoring application and its development process must adhere to must be stated. Fulfilment of these criteria affirms the belief in the claims stated in the previous section. The criteria will be discussed analogous to the three identified sub-claims iterated in Section \ref{sec:claims}.

Again we re-iterate, these criteria and requirements only relate to the validation study, not the requirements of the actual monitoring application prototype that will be designed and developed. Reason for this is that the aim is to evaluate the development platform, not this particular instance of the platform.

\subsubsection*{Applicability}
Intuitively, the primary criterium is that an instantiation of the proposed platform should be possible in accordance with the needs and wants of \idsystems . This seems an obvious and trivial demand, but without stating it, any subsequent criterium is pointless. More specifically, the platform should enable an instantiation which enables iterative and consequent enrichment and accumulation of information. At multiple stages of the consequential iteration the application should be able to generate outputs such as alerts and reports for auxiliary processes and systems. This requirement validates the applicability claim of Section \ref{sec:claims}.

\subsubsection*{Usability} 
Though the platform should enable an instantiation according to the needs of \idsystems, it should do so with minimal development effort. These efforts will be expressed in the number of code instructions required to realize the implementation. Since an absolute benchmark was difficult to ascertain, the upper bound of permissible number of code instructions is established relative to the amount of instructions necessary for a functionally similar monolithic implementation. Should a larger code-base be determined, this entails a level of abstraction that is too low and requires (repeated) implementation of procedures that should have been provided by the platform itself. For the construction of the topology it was chosen to allow at most 4 operations for every component in the platform topology. The parameter of 4 operations per component originates from an assertion made in Chapter \ref{ch:architecture}.

\subsubsection{Adaptability}
For the adaptability of the application provided by the platform it was determined that minor new features and requirements should require not more than
\begin{itemize}
\nospace
\item a localized rearrangement of the model/topology, 
\item introduction or major change of at most two components, 
\end{itemize}
As for all cases, small changes are allowed to the components interfacing with the altered component(s) in order to produce or consume information supplied to or emitted by the altered component. Additionally, we allow for very minor, consistent changes to be made to other components. The reason for this is that often a change or introduction of a datapoint requires that change to be forwarded throughout the topology.

The rationale for these allowances is that the modularization provided by the platform should prevent entanglement of concerns and therefore minor changes should cause localized effects. There is however a possibility that (especially new features) require a change in several components since its the functionality was not previously present. Therefore minor consistent changes are allowed to those components in order to forwarded the new functionality. Finally, the reason for the allowance of a major change in two components is that often computation and analysis of a datapoint is separated into distinct components due to separation of concerns. Therefore a changed requirement will often require a change in both components.

\subsubsection{Summary of criteria}
The concrete criteria formulated in this section are as follows:
\begin{enumerate}
\nospace
\item An instantiation of the proposed platform should be possible in accordance with the needs and wants of \idsystems, which allows for
\begin{itemize}
\item iterative and consequent enrichment and accumulation of information, and
\item output consequences at multiple stages of computation.
\end{itemize}
\item The instantiation of the platform should require 
\begin{itemize}
\item no more calculation code than it would in a monolithic system, and
\item at most 4 instructions of code per component to build the topology.
\end{itemize}
\item A minor change in the goals or requirements should require only one change in the topology, a single component (and its interfacing) or a small and consistent change in multiple components.
%\item A realistic deployment of the instantiated application should be able to handle an input of 100.000 devices.
%\item A reconfiguration of the realistic deployment should be able to handle a hypothetical load of 1 million input devices, without changing the topological order of the components.
%\item An event burst of factor 100 should be processed within [number] seconds.
\end{enumerate}

%Aside from the usability criterium we find the scalability required of the platform. We will formalize this requirement with two criteria. The first regards the general scalability of the platform. It requires the application to be able to cope with fantastic amounts of input data by only reconfiguring the worker tasks of the application, without changing the topological order of those components. Secondly, the implementation should be able to cope with fluctuating data signatures. For this the following criterium was formulated. The platform implementation should return to normal execution parameters within a certain amount of time after experiencing an increased input load.

%\subsubsection{Criterium parameters}
%While the functional criteria pose a binary decision on pass or fail, the non-functional criteria require quantification in order to determine weather they hold for the platform implementation. These parameters are based upon contemporary and near-future use cases and have been determined in collaboration with industry experts of \nedap. For the usability requirement it was chosen to express them in the number of instruction required. As an absolute upper bound on the amount of code proved to be difficult to determine before-hand, it was defined as "the amount of instructions and computations required for calculating the QoS parameters in a monolithic application, plus at most 4 lines of code for every component in the platform topology". The parameter of 4 lines of code per component originates in assertion made in Chapter \ref{ch:architecture}.

%For the adaptability of the application provided by the platform it was determined that minor new features and requirements should require not more than
%\begin{itemize}
%\nospace
%\item a localized rearrangement of the model/topology,
%\item introduction or major change of at most two components, 
%\end{itemize}
%As for all cases, we allow small changes to the components interfacing with the altered component(s) in order to produce or consume information supplied to or emitted by the altered component. Additionally, we allow for very minor changes to be made to other components. The reason for this is that often a change or introduction of a datapoint requires that change to be forwarded throughout the topology.

%The rationale for these allowances is that the modularization provided by the platform should prevent entanglement of concerns and therefore minor changes should cause localized effects. There is however a possibility that (especially new features) require a change in several components since its the functionality was not previously present or necessary. Finally, the reason for the allowance of a major change in two components is that often computation and analysis of a data is separated into distinct components due to separation of concerns. Therefore a change requirement will often require a change in both components.

%Finally, for the scale of the input signature for the monitoring application, the realistic near-future scale of the sensor application was determined to be 100.000 devices. For the fantastical size of future applications we have taken an increased factor of $\times$10, i.e. 1 million devices. Finally, for the increased signature of the temporary event shower to be processed we have chosen a factor of $\times$100 of the realistic data signature. This burst is supposed to be processed within one minute, after which the application should return to regular execution parameters.




%functional
%	features
%	not measurable, just attainable
%	- multilevel reporting
%	- enrichment through platform
%non-functional
%	scale 
%		x sensors
%	ease-of-implementation
%		1 fte week
%	measurable
%	validation criteria
	
\section{Implementation of the WSN monitoring application}
\subsection{Design and Implementation}
In this section the design for the \nedap \sensit and its implementation details will be described. We will first take a top-down look at the entire topology. After which we will shortly describe the functionality of the individual components.

\subsubsection{Application topology}
The designed topology is depicted in Figure \ref{fig:sensit_topology}. From this we see that the processing is divided into three stages. In the first stage raw-information snapshots are enriched and normalized. In doing so it improves the information potential and accuracy of the data in the snapshot. The second stage concerns sensor level analysis and management. It calculates the state and resource consumption of the devices, and it includes some services that alert if a sensor exhibits abnormal behaviour or long term deviations of its ordinary parameter margins. The final stage concerns snapshot accumulation in order to extract high level information and decisions from it. This stage diverges into three distinct accumulator paths. The top path performs accumulations of snapshots based on the sensor group ID. It reports on data rate violations (as agreed upon in SLA's) and recalculates the share of the data each sensor within a sensor group is allowed to consume. The middle execution path concerns the cells served by nodes. It alerts if a node switches cells more then an allowed amount during a period. The bottom accumulates all the snapshots in order to report on the current state of the application as a whole.

\begin{sidewaysfigure}
\centering
\includegraphics[width=1.15\textwidth]{resources/img/sensit_topology.png}
\caption{Topology of the monitoring application for the \idsystems \sensit WSN application}
\label{fig:sensit_topology}
\end{sidewaysfigure}

We will conclude the description of the application topology by shortly describing the functions of the individual components.
\begin{description}[style=nextline]
\nospace
\item[Sensit spout] Reads sensor snapshots from a Kafka channel and introduces them into the topology.
\item[Translator] Translates the sensor information from hexadecimal string to key-value pairs.
\item[Nuancer] Averages the data points received from a sensor to eliminate abnormalities. It does so by keeping a record of the last seen messages for each sensor node in an SQL database.
\item[Attributor] Enriches the snapshot with some datapoints not present in the sensor but known by  back-end applications.
\item[Sensor RDM processor] Processes the enriched information from the snapshot and calculates the optimal operational device strategy.
\item[Switch strategy buffer] Buffers the switch strategy messages to prevent superfluous, erratic feedback to the sensors. Doesn't switch strategy on first report, only if a switch is requested over an extended period.
\item[Single message analyser] Calculates weather the sensor parameters (as calculated by the RDM processor) are within the allowable margins.
\item[Budget recalculator interface]If the message rate of a sensor is high enough will initiate an immediate budget recalculation. If message rate is low it is allowed to be accumulated over some time to reduce the number of database updates.
\item[Budget recalculator accumulator] Accumulates budget recalculation snapshots and prepares them  for batch update.
\item[Budget recalculator] Executes batch budget recalculation.
\item[Group accumulator] Accumulates snapshots by sensor's group ID. Because this is performed on a weekly basis, this is performed two-stage as not to cause a large data build-up over time.
\item[Group share recalculator] Recalculates the share of the sensor group's resources each sensor is allowed to consume, based on the data used by each node over a one week period.
\item[Application accumulator] Accumulates the information emitted by the application in order to be presented on an application dashboard.
\item[Cell Switch Analyser] Analyses and reports if a node switches between cell towers more then is allowed.
\end{description}

Final remark on the application design is on the interfaces it provides. The application's inputs and outputs received and provided to Apache Kafka channels. This allows actual services to be easily swapped in and out with test services (even at runtime). 

\subsubsection{Sensor Resource Distribution Model}
To model the state, behaviour and strategies of the sensor we employed the RDM model proposed in Chapter \ref{ch:rdm}. The resulting model is depicted in Figure \ref{fig:sensit_rdm}. The model takes a few parameters based on the sensor state measurements, such as its current ECL and message rate, and its history, such as its runtime, data already used and budget already used. The model then computes the runtime the sensor has left, current data and budget consumption and the optimal mode of operation. By optimal we mean the operational strategy with the highest message rate that will not exceed the resource availability.

\begin{figure}
\centering
\hrule
\includegraphics[width=\textwidth]{resources/img/sensit_rdm.png}
\hrule
\caption{Resource Distribution Model for a sensor in the \idsystems \sensit WSN application}
\label{fig:sensit_rdm}
\end{figure}

Earlier experiments with resource consumption models have shown that a device will act differently when in the beginning than in the end of its life-cycle, when there is a scarce resource involved. The reason for this is that in the beginning the models will instruct the device to operate on a strategy that will consume less resources then it is allowed on average. Then, when it has saved up enough of that resource, it is allowed to spend it on a strategy that consumes more than that average. To mitigate this effect is has been chosen to recalculate the available resources on a monthly basis. This way there is still such a cycle, but its period is far shorter and the effect will be much less and much more regular overall.

\subsection{Adapting the application}
We will conclude this section by deliberating some hypothetical adaptations in order to investigate the tunability of the platform.

\subsubsection{Nuancer local}
The first change we introduce is the constraint for the \emph{Nuancer} to not require a database connection. Reason for such a requirement is to reduce latency or to eliminate capacity issues caused by employing an SQL database. 

This can be achieved by exchanging the current \emph{DatabaseBuffered} Nuancer implementation with a \emph{SingleMessageProcessor}. This processor keeps an in-memory cache of the last snapshots it has encountered, grouped by node and ordered by timestamp or sequence number. For each incoming snapshot the following sequence of actions is taken:
\begin{enumerate}
\nospace
\item determine node by ID,
\item add snapshot to the node's buffer,
\item purge out-of-scope snapshots from the cache,
\item calculate average of remaining buffered snapshots, and
\item emit averaged snapshot
\end{enumerate}
This sequence of actions is similar to how the Nuancer operates in the current topology, but it eliminates the database connection in favour of a local buffer of snapshots. Unfortunately by shifting to a local buffer we can no longer employ the scaffolding provided by the \emph{DatabaseBufferComponent}. The reason for this is that the component with local buffer (as currently implemented) operates on a single global buffer, instead of a buffer per node.

Finally, it must be noted that in requiring the snapshots to be cached locally, a large burden is forced upon the memory of the machine running the component. Should the application serve a large amount of nodes and snapshots are collected within a large window if interest, the data kept in-memory can rapidly reach large sizes. This can be alleviated by replicating this component to the point that individual memory requirements of workers are within manageable parameters. Alternatively, the memory issue can be evaded by persisting and reading snapshots to local files. This introduces some latency due to disk IO, but can immensely reduce the number of records in the active cache at any time.

\subsubsection{New sensor data encoding}
As mentioned, the auxiliary performance data of the sensor is received as an encoded hexadecimal string. For this case we introduce a new type of sensor equipped with a different radio module, which encodes its performance datapoints slightly differently. Though deliberated as a hypothetical, this case simulates a real future scenario. Since the aim is for a node lifetime of at least 10 years, it is very conceivable that sensor wireless technologies improve and change during that timeframe. Since physical replacement of the large volumes of deployed nodes is unprofitable for both \nedap and its clients, this new technology should be supported in tandem with the old sensor types. We emphasize that for this case we do not significantly change the actual data collected and emitted by the device as this would entail a major change in how computations need to be performed.

This change in the sensor environment can be accommodated for by introducing a second \emph{Translator} component specifically intended for the new data format. This component is executed independently of, and in parallel to, the original Translator. How to ensure that a snapshot is processed by the correct translator, will depend on how the new the data stream is supplied to the application. For this hypothetical we will consider the most complicated input option, where the old and the new style snapshots are emitted on a single input channel. To split the singular input stream into two an interface component is introduced. This component performs a superficial inspection of the snapshot and forwards it to the correct Storm channel based on some discerning feature (format, type identifier, etc.). Though technically this inspection could be performed by the \emph{SensorSpout}, separation of concerns compels a separate component for this purpose. Subsequently, both translators uniformly emit their translated snapshots to a common Storm channel for further processing. The resulting partial topology is illustrated in Figure \ref{fig:update_encoding}

\begin{figure}
\centering
\includegraphics[width=0.7\textwidth]{resources/img/update_encoding.png}
\caption{Updated partial topology for new data encoding}
\label{fig:update_encoding}
\end{figure}

\subsubsection{Alert on long-term ECL drop}
For our final case we extend the functionality of the application by introducing a new outcome for the application. The added requirement is the detection of long term drop in ECL level. Such a drop could signify a, possibly alleviable, obstruction placed between the sensor and the sensor sink. Moreover, should several geographically related sensors report such a disruption, drastic actions cannot be ignored. In our topology this can easily be achieved by extending one of the existing components. Formally, the \emph{CellSwitchAnalyser} would be most suited for this purpose, since it is already historically aware due to retaining a list of cell towers per sensor. Though the component would obviously require renaming.

We provide this functionality by keeping a list of ECLs reported by each sensor node. When the sightings are inconsistent or do not feature a drop, the list is purged. When the list's size (or timestamp difference) surpasses a set threshold, an alert is sent to the alerter. This is easily performed since the CellSwitchAnalyser already features alerting functionality. Finally this change does not require changes to interfacing components, since the ECL level is already present in the snapshot emitted by the \emph{SensorRdmProcessor}.

\section{Results}
We will report the results under their own three headings: applicability, development effort and adaptability

\subsubsection{Applicability}
The sensor model was found to be adequate for modelling the behaviour of the \sensit sensors. The modular design proved very useful for expositioning the different resources and how they were interconnectively calculated and distributed. Unfortunately (for the purpose of this study), the sensor did not feature a large variety of resource metrics specifying its configurable behaviour and therefore the model only featured one configurable component. Additionally, after accumulation of the application-level parameters by the \emph{ApplicationAccumulator} the accumulated parameters needed no further transformations and the WSN application did not feature application-level configuration needs. Therefore the Resource Distribution Model was only employed on the sensor-level.

The result of the applicability investigation with regard to the distributed topology is that the platform suffices as development platform for the purposes of \idsystems. The building blocks provided enable the implementation of a functional application and provides functional abstraction of the specifics of the underlying technologies.  During implementation of the application it was noted however that the platform does not provide an efficient way of buffering and processing snapshots grouped per node, cell tower, etc. 

\subsubsection{Usability}
%It took about 80 hours to conceive, implement, debug and tune this architecture implementation. This time spent can be divided into roughly 15\% design, 35\% initial implementation and 50\% debugging and tuning \footnote{All hours spent on a component after its first inclusion in the topology and executing it are pooled into the latter category}.
%TODO model requirements res+comp+interf+rdm*connRes
Specifying the sensor model and application topology could be performed within the set parameters. As claimed, each component requires but four actions to be introduced to the topology. These actions are create component, declare component, subscribe as consumer, declare output channels.

However the internal code of the topology components, which actually performs the calculations and computations, required about twice the amount of code that a monolithic application would. While the actual number of lines of code was only a little higher than then its monolithic counterpart would, the computations and transformations performed on those lines was far more then would be necessary in a monolithic application. These discrepancies will be deliberated on further in Section \ref{sec:eval}.

\subsubsection{Adaptability}
Finally, the necessary adaptations to the existing application for each hypothetical case are summarized in Table \ref{table:adaptations}.

\begin{table}
\centering
\begin{tabular}{|l||c|c||c|c|} \hline
Summary				& \multicolumn{2}{c||}{Components}		& \multicolumn{2}{c|}{Topology changes} \\ 
					& new 	& changed 	& none 		& only local  \\ \hline 
Nuancer local		& 0		& 1			&			& \cmark \\ \hline
New sensor encoding	& 2		& 0			& 			& \cmark \\ \hline
Alert ECL drop		& 0		& 1			& \cmark	&		 \\ \hline
\end{tabular}
\caption{changes required per adaptation scenario}
\label{table:adaptations}
\end{table}

%Applicability
%	goed, mist mapped-buffer
%2x zolang als gepland
%	tracability difficult
%	disjunction 2d model, 1d coding
%	debug first run duurt lang want string-bindings en string serialized
%	part testing is goed (dump result/intermediate)
%	solved by structs
%Meer code dan gepland (bijna 2x zovel)
%	building models en topology wel goed
%	(de)serializing
%	solved by structs
%[todo:timing/scalability]
\section{Evaluation}
\label{sec:eval}
In this section we will evaluate the obtained results and compare them to the criteria set out in section \ref{sec:criteria}. The criteria will be deliberated in the same order as the results in the previous section were.

\subsubsection{Applicability}
As stated, the building blocks provided by the platform allowed for a sufficient implementation of the intended monitoring application. However it was discovered that, though possibly useful, the platform did not provide an efficient template to buffer snapshots grouped by a certain snapshot parameter.  This component could easily be provided by introducing a mapper function to the \emph{BufferedProcessor} which will determine into which buffer a snapshot will be added to. The existing filter, sort and execution methods will then be performed on these buckets individually, providing a mechanism of grouped computations.

However such functionality currently is not present, this absences was easily avoided and was found to be only a minor inconvenience. Since this issue singularly was not sufficient to invalidate the applicability criterium we state that Criteria 1-3 hold.


\subsubsection{Usability}
%As mentioned it took about 80 hours to construct an develop the prototype application. This is twice as many as was originally stated in the validation criterium (Criterium 4). 
As mentioned in the results, though the code required to compose the resource models and application topology was contained within the specified parameters, the code required for the internals was found to be significantly more than a monolithic application would require. We therefore yield that \emph{usability} criterium was invalidated. The chief reason that the component internals required  more instructions as required in a monolith is the repeated serialization an deserialization of data into messages. Processing of each (group of) snapshot(s) is prepended with a few lines of code that extract, parse and cast each individual variable from the snapshot. After the component's processing is performed, a new snapshot is prepared with variables that again require its values to be serialized. When the computations of a component only amount to a few lines of code, this (de)serialization can quickly require more code than the actual computations do.

This was also reflected in the time required to develop this prototype. It was initially expected that the instantiation could be constructed within 40 man hours. However, this eventually took twice as many hours. Of that time about 15\% was spent designing, 35\% developing and 50\% debugging the application\footnote{All hours spent after fully constructing and first execution of the application are pooled into the latter category}. When we inspect the breakdown of the time spent we yield that it took an enormous amount of time to debug and adapt the components after its original design and implementation. The chief reason for this was found to be the loose coupling between components. The components are completely disjoint and the snapshot variables they share require custom serialization in between components and accessed with string identifiers. This entails that it is excessively easy to implement a broken component. This is since inappropriate variable access due to misspelled identifiers can occur very easily and is not detected by code checkers and compilers of conventional IDEs. Subsequently, when the variable is accessed successfully, the value often requires deserializing into the correct primitive or object type. This again introduces a possible point of failure due to misparsing and miscasting, since the compiler cannot detect the actual object type without executing the application.

To alleviate both the above mentioned problems we propose the introduction of snapshot struct objects (POJOs). These objects contain the variables of the snapshots passed between components. However, in contrast to loosely coupled key-value bindings, these bindings are explicitly defined in both type and identifier. They can therefore easily be serialized and deserialized by common serialization mechanisms. This would aliviate the need for developers to continually specify custom serialization. By providing direct access to the correctly parsed variables in the snapshots it will reduce the code base by a huge amount. Additionally, by providing a mechanism to directly access the properly parsed variables, the number of possible instances where mismatching, misparsing and miscasting can occur is reduced. Thereby eliminating several points of possible failure which have proved problematic. Combined, this increased traceability and automated (de)serialization should have a noticeable, positive effect on the amount of required code and the time spent debugging and reworking the application, and thus the development time as a whole.

To illustrate this benefit, two simplified code snippets from the \emph{SensorNuancer} are presented. One which does not employ structs (Listing \ref{list:nuancer_without_structs}) and one which does (Listing \ref{list:nuancer_with_structs}). From these examples it is clearly observable that by employing well-defined, serializable structs we are able to reduce the instruction required due to serializing and deserializing, and it reduces the chance of mismatching variable identifiers by eliminating string bindings.

\begin{scriptsize}
\begin{lstlisting}[
language=java, 
caption={Simplified fragment of \emph{SensorNuancer} without struct objects}, 
label={list:nuancer_without_structs}, 
escapeinside={(*}{*)}, 
captionpos=b,
numbers=left,
tabsize=4
]
public void runForMessagesHistoric(LinkedList<IOMessage> history) {
	Map<String, String> args = new HashMap<>();
	long first = Long.parseLong(
		history.getFirst().getVars().get("TIMESTAMP"));
	long last = Long.parseLong(
		history.getLast().getVars().get("TIMESTAMP"));
			
	List<Integer> ecls = new LinkedList<>();
	for(IOMessage m : history){
		ecls.add(Integer.parseInt(m.getVars().get("ECL_LOCAL"));		
	}
	int normalizedEcl = normalizeEcl(ecls);
			
	args.put("MILLIS_ELAPSED", Long.toString(last-first));
	args.put("ECL_LOCAL", history.getLast().getVars().get("ECL_LOCAL"));
	args.put("ECL", Integer.toString(normalizedEcl));
	publish("SENSOR_NORMALIZED", new IOMessage(args));		
}
\end{lstlisting}
\begin{lstlisting}[
language=java, 
caption={Simplified fragment of \emph{SensorNuancer} with struct objects}, 
label={list:nuancer_with_structs}, 
escapeinside={(*}{*)}, 
captionpos=b,
numbers=left,
tabsize=4
]	
public void runForMessagesHistoric(LinkedList<NuancerInStruct> history) {
	NuancerOutStruct output = new NuancerOutStruct();
	long first = history.getFirst().getTimestamp();
	long last = history.getLast().getTimestamp();
			
	List<Integer> ecls = new LinkedList<>();
	for(NuancerInStrcut struct : history){
		ecls.add(struct.getEclLocal());
	}
	int normalizedEcl = normalizeEcl(ecls);		
			
	output.setMillisElapsed(last-first);
	output.setEclLocal(history.getLast.getEclLocal());
	output.setEcl(normalizedEcl);				
	publish("SENSOR_NORMALIZED", output);	
}
\end{lstlisting}
\end{scriptsize}

Finally, it was noted that after initially specifying the topology and models reworking them proved to be frustrating. The difficulty was mainly in locating the instantiation and declaration of a component in the code that builds the topology. The reason for this is that it constantly requires a developer to transition from a two-dimensional graphic image of the model or topology to builder code which is one-dimensional (top-to-bottom). This mental transition can be avoided by eventually developing  graphic development tools that allows a developer to conceive a topology by drawing a graphical model of components and resources. The appropriate computational code can then later be introduced into the components. By doing so a developer would only need to concern themselves with one depiction of the topology instead of two.

\subsubsection{Adaptability}
From Table \ref{table:adaptations} we find that all three scenarios conform to the set criteria. All minor changes to the requirements context were incorporable with the existing application by introducing or changing at most two components. Additionally the adaptations require either no changes to the topology or only small, localized changes. Incidentally, these scenarios required no changes to the components interfacing with the changed or introduced components.

%applicability passd, with small notion
%devtime -> fail
%	tracability difficult
%	disjunction 2d model, 1d coding
%	debug first run duurt lang want string-bindings en string serialized
%	part testing is goed (dump result/intermediate)
%	solved by structs
%timing/scalability TODO
\section{Discussion}
We will conclude this chapter by contemplating on the outcomes. First, we will state the conclusions drawn from the performed study. Secondly we will discuss the validity of the study and therefore the conclusions drawn. We will conclude by deliberating limitations of this short validation study.
\subsection{Conclusions}
The main conclusion to draw from this initial validation study is that it indicates the development platform to be a functional tool to develop a functional WSN monitoring application. The distributed application architecture provides a functional separation of concerns and the provided component scaffolding provide curtailment of most types of data streams and distributions. Secondly, the explicit Resource Distribution Model provides a useful exposition of how resources within a system are interconnected, calculated and utilized. Additionally, the explicit nature of the model allows unknown variables to be computed in accordance with the model's constraints and optimal behaviour.

This study has shown that, for the purpose of the \idsystems \sensit application, the monitoring solution can be constructed within the set parameters for required development effort, with the exception of the required implementation of component's internals. Additionally, the provided capability for separation of concern allows for rapid software evolution within the context of minor changes to the monitoring application's requirements or context. There are however some small deficiencies and issues to be solved in order to also make the platform more practicable. 

The first main issue to be resolved is the inclusion of functionality to buffer snapshots grouped by some parameter(s) of those snapshots. The second issue regards the inclusion of structs (POJOs) to be used to communicate between components. These structs can be automatically serialized and deserialized and they increase the traceability of datapoints between components. This will reduce the code and time required for development. It might be argued that these structs themselves will introduce new code to the application. However these objects are easily generated by conventional code generators. This approach will therefore reduce the overall development effort required. As the components will no longer be disjunct, but linked by these objects, it will reduce the time spent debugging the application significantly.

Secondly, the inclusion of a graphical model/topology editor will remove the disjoint between graphical design documents and actual implementation. This will further reduce the development effort as a developer is no longer required to transition constantly between two representations of the developed artefacts.

\subsection{Discussion}
To solidify the validity of this study, some contending issues must be addressed. 

\subsubsection{Representativeness of the \sensit application}
The first issue of which is the applicabilty of the study. For any assertion to be relevant to the field of LWPA WSN it must be demonstrated that the \sensit application is representative and conforms to the characteristics for LWPA WSN applications. Table \ref{table:lpwa-chars} lists the typical LPWA WSN characteristics, as reported by multiple sources \cite{lora-vs-sigfox-boek, lora-vs-sigfox-whitepaper, lora-vs-nbiot-vs-sigfox, lora-vs-sigfox, whitepaper-tmobile, nbiot}.

\begin{table}
\centering
\begin{tabular}{|l|l|}\hline
Characteristic & Value \\ \hline
Message payload & $<$ 256 Bytes	\\ \hline
data rate &	~1.6 KiB/day/node \footnote{Actual objective universal message/data rate bounds are difficult to obtain, since different technologies prioritize varying limiting factors (message rate, data rate, energy consumption, etc.)}  \\ \hline
node lifetime & 10 years \\ \hline
node costs & $~5$\$ \\ \hline
Network infrastructure & Star topology (cellular)	\\ \hline
\end{tabular}
\caption{Characteristcs of typical LPWA WSN applications}
\label{table:lpwa-chars}
\end{table}

From the table summation and the application parameters stated in Section \ref{sec:sensit} we conclude that the \sensit application conforms to the typical features of LWPA WSN applications. Intuitively, the node costs and lifetime, 5\$ and 10 years respectively, match the parameters typifying LWPA applications. Additionally, SENSIT's new NB-IoT network technology features the typical cellular star topology. More importantly, the LWPA data signatures encompass the data signatures featured by the \sensit application. The [100] Bytes per message are well contained within the [typical] maximum of 256 Bytes. Finally, supposing a message rate of 15 message per day and a payload of [100] Bytes per message yields a daily per sensor data rate of about 1.5 KiB. Though the actual daily message rate of a node can vary wildly, as do the general bounds for individual network technologies, the averaged rate conforms to the approximated per sensor data rate typical of LWPA WSN applications.

\subsubsection{Threat of over-abstraction}
As mentioned, the current state of the development platform features some deficiencies. Should these aforementioned deficiencies be absolved and the new functions provided, the level of abstraction is raised. Therefore it must be ensured that the level of abstraction is not raised to the point that the applicability claim (sub-claim 1) is invalidated. For the inclusion of a \emph{MappedBufferedProcessor} this concern is trivial as it provides an abstraction but, as it is extends to the platform, it does not obfuscate any underlying functionality. In selecting or implementing a serialization mechanism, note should be taken that it can transform every innate or user-specified datatype. Provided this concern is considered, a higher level of abstraction is provided, but no functionality is lost. Finally, the to be included graphical modelling/development interface should allow definition, specification and interconnectivity between all components provided by the platform. To this end, it is urged that the graphical interface is included in the platform instead of developed alongside the platform as a separate project. Separate project development will inherently lead to the development of the graphical interface trailing the development main platform and possible diverging of goals and requirements. If curtailment of all the above mentioned concerns is guaranteed, the level of abstraction can be raised to an appropriate level while safeguarding the applicability claim.

\subsubsection{Developer skill level}
a final point of contention regarding the validity of this study is the subjectivity of the executor. The study was performed by a subject with full knowledge of the internals of the development platform. Though this allows for rapid development and exploration of the capabilities of the platform, it possibly undermines the conclusions made on required development effort. Reason for this is that the actual subject may be over-skilled with regard to a representative developer of a QoS monitoring application. Therefore care must be taken that the general development effort is not underestimated. The likelihood of such an underestimation will be deliberated in this section.

Firstly, we will deliberate the construction of Resource Distribution Models. Though this study does not assert bold claims regarding the effort of constructing such models, we can predict and discuss the relative impact of a reduced skill level to the effort required. Though a model instantiation may seem daunting, it is actually constructed using only a few concepts. A model consists of \emph{Resources} and \emph{Components} computing, consuming and producing these resources. Respectively, components are connected to resources by an interface of type \emph{Calculates}, \emph{Consumes} or \emph{Produces}. The only issue complicating this depiction is the \emph{ModelledComponent}, which contains multiple utilization models with a resource interface for each resource interfaced by the component. However, these interfaces are instantiated and act equal to the regular component-resource interfaces. Therefore understanding of one carries over to the other. Finally, specifying the intended model may prove challenging to less familiar developers. This is due to the nature of the formula specification of resource interfaces. These formulas are very formalized to enable automated computation and evaluation of instantiations. These interface formulas take an array as input containing all input values required to compute its output. Consequently, a list of resource identifiers is provided to the function, specifying the resources to be inserted at each index of the input array. In doing so it provides a compact specification for these formulas. However, it also allows for construction of invalid, incalculable or semantically incorrect models. Therefore clear and indubious instructions will be provided to guide future developers.

Finally, we consider the consequences to the application topology. Firstly, the internals of the topology components are plain Java code. Therefore the level of familiarity has a negligible effect to implementation of the internals. Secondly, the suggested introduction of a formalized and automated (de)serialization will only aid an uninformed developer, since it provides a clear handle to the implementer, obfuscating the cumbersome details of the underlying communication platform. Additionally, the construction of the application topology was concluded to be specifiable by four instructions per topology component. The skill level of the application developer/designer has no impact to this required number of instructions, since the provided \emph{TopologyBuilder} contains no actions aside these four instructions for a component: create component, declare component, subscribe to channels, declare as producer to channels. 

Finally, we argue that an unskilled implementer will gain more form the platform then the acquainted subject which performed this study. This is asserted due to the limited number of component types that require understanding. The platform only features five different types of components, with at most two variations per component (e.g. distributed/local computation or database/local buffer). Additionally, the scaffolding provided will help developers in specifying more complicated components. For example the \emph{DatabaseBufferedComponent} requires implementations for abstracted methods that subsequently \emph{add to}, \emph{fetch} and \emph{filter} the buffer managed by the database. This sequence specification guides a developer in implementing the intended behaviour of the buffer. Therefore, we argue that a less skilled developer will gain more benefit from the platform, relative to his/her skill level.

\subsection{Limitations and recommendations}
%TODO repeated execution (different applications) cements claims and boundries
Though this validation study demonstrates the platform to be a useful tool, it must be regarded as a proof-of-concept. This study only regarded one sensor application and therefore the results might be accidental and therefore the evidence provided by them is highly anecdotal. Though the preliminary results do indicate the platform to be a useful tool for WSN QoS monitoring, general statements are not allowed to be asserted unequivocally regarding the general applicability of this tool to the field of WSN applications. For such conclusions to be asserted, much more validation on a more varied base of applications is required.

A second shortcoming of this study is that the \sensit wireless sensor application did not feature the complex cases to fully explore the capabilities of the Resource Distribution Model. Previous chapters have claimed that the Resource Distribution Model should be applicable at multiple stages of information processing (e.g. sensor, per cell, entire application). However, as mentioned before, there was no case for post-accumulation processing or sensor configuration based on application-level parameters. Therefore no RDM was employed in the latter stage of information processing. We therefore only claim the model to be applicable at sensor level for the \sensit application. In order to assert the model as a general solution, more research should be performed on sensor applications that do feature more complex application level processing or configuration.

Procedurally, this study also features a large limitation and therefore so do the conclusions drawn from it. The limitation to the study is that it was not designed as a blind study. As the application instantiation of the platform was developed by a developer with full knowledge the validation criteria and intimate knowledge of the internals of the development platform. In order to fully and objectively assert the conclusions of this study the experiment must be repeated more formally with impartial subjects. These subjects must be able to repeat the experiments process without knowledge of the parameters of the study, without familiarity of the platforms internals and only the provided documentation of the platform and its exposed APIs.

We however propose that this eventual full-scale study is not performed until the latter stages of platform development and validation. The reason for this is that it is far more resource-efficient to discover initial deficiencies and issues with small case studies, as performed in this chapter. Only when these studies no longer yield suggested improvements to the platform should the scope be focussed towards more expensive, formalized studies.
\chapter{Conclusion}
\section{Discussion}
why not sensor or edge computing
feedback into models (learning models)
wider applicability?
\section{Conclusions}


\bibliography{bibliography}
\bibliographystyle{ieeetr}
%timestamp - time-stamp
%datapoint - data-point - data point
%criterium - criterion
%proofed -> proved
%proof of conecpt - proof-of-concept
%building block - building-block
%bytes -> Bytes
%lwpa -> lpwa
%telcom - telecom
%large scale -> large-scale
%big data - big-data
%backtrack search -> backtrack-search
%data set - data-set - dataset
%micro-service -> micro->component
%purge -> prune
%aggregator - accumulator
%RabbidMQ -> RabbitMQ
%re* -> re-*
%weather -> whether

%TODO capitalization
%state of the art/affairs
%rum/rdm

%rum/rpm/rdm
%koppelzin \comma\ daadwerkelijke zin
%cite before dot
%processor names consequent
%woordenlijst/afkortingen?
%*-level

\end{document}
